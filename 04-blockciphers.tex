\documentclass[11pt]{article}
\usepackage{fullpage}
\usepackage{amsthm}

\usepackage{amsthm,amsmath,amsfonts,amssymb,amstext,enumitem}
\usepackage{latexsym,ifthen,url,rotating,graphicx}
\usepackage{listings}
\usepackage{tikz}
\usetikzlibrary{arrows,shapes,positioning,fit}
\usepackage{graphicx}
\usepackage[font=small,labelfont=bf]{caption}



% --- -----------------------------------------------------------------
% --- Document-specific definitions.
% --- -----------------------------------------------------------------
\lstset{
    columns=fixed,
    literate={—}{{---}}1 {…}{{...}}1
}

\newcommand{\todo}[1]{{\color{red}[TODO:{#1}]}}

\newtheorem{problem}{Problem}
\newtheorem{corollary}{Corollary}
\newtheorem{fact}{Fact}
\newtheorem{exercise}{Exercise}
\newtheorem{theorem}{Theorem}
\newtheorem{definition}{Definition}
\newtheorem{notation}{Notation}
\newtheorem{lemma}{Lemma}
\newtheorem{example}{Example}

\newcommand{\getsr}
  {{\:\stackrel{\raisebox{-2pt}{${\scriptscriptstyle \hspace{0.2em}\$}$}}
   {\leftarrow}\:}}
\newcommand{\points}[1]{\textbf{({#1} pts)}}

\newcommand{\Colon}{\ : \ }
\newcommand{\st}{\mathsf{state}}
\newcommand{\msgs}{\mathcal{M}}
\newcommand{\ctxts}{\mathcal{C}}
\newcommand{\keys}{\mathcal{K}}
\newcommand{\kg}{\mathcal{K}}
\newcommand{\enc}{E}
\newcommand{\dec}{E^{-1}}
\newcommand{\MAC}{\mathrm{MAC}}
\newcommand{\RMAC}{\mathrm{RMAC}}

\newcommand{\pk}{pk}
\newcommand{\sk}{sk}

\newcommand{\calD}{\mathcal{D}}
\newcommand{\AES}{\mathsf{AES}}

\newcommand{\algorithm}[1]{\textbf{Alg} {#1}}

\newcommand{\calO}{\mathcal{O}}

\newcommand{\dlog}{\mathrm{dlog}}

\newcommand{\Adv}{\mathbf{Adv}}
\newcommand{\AdvPRF}[2]{\Adv^{\mathrm{prf}}_{#1}({#2})}
\newcommand{\AdvPRG}[2]{\Adv^{\mathrm{prg}}_{#1}({#2})}
\newcommand{\AdvCPA}[2]{\Adv^{\mathrm{ind{-}cpa}}_{#1}({#2})}
\newcommand{\AdvCCA}[2]{\Adv^{\mathrm{ind{-}cca}}_{#1}({#2})}
\newcommand{\AdvKR}[2]{\Adv^{\mathrm{kr}}_{#1}({#2})}
\newcommand{\AdvCKR}[2]{\Adv^{\mathrm{ckr}}_{#1}({#2})}
\newcommand{\AdvRMR}[2]{\Adv^{\mathrm{rmr}}_{#1}({#2})}
\newcommand{\AdvCR}[2]{\Adv^{\mathrm{cr}}_{#1}({#2})}
\newcommand{\AdvUFCMA}[2]{\Adv^{\textrm{uf{-}cma}}_{#1}({#2})}
\newcommand{\AdvDL}[2]{\Adv^{\mathrm{dl}}_{#1}({#2})}

\newcommand{\Exp}{\mathbf{Exp}}
\newcommand{\ExpOW}[1]{\Exp^{\mathrm{ow}}({#1})}
\newcommand{\ExpCKR}[2]{\Exp^{\mathrm{ckr}}_{#1}({#2})}
\newcommand{\ExpRMR}[2]{\Exp^{\mathrm{rmr}}_{#1}({#2})}

\newcommand{\concat}{{\,\|\,}}
\newcommand{\xor}{\oplus}
\newcommand{\bits}{\{0,1\}}

\newcommand{\tcolh}{T^{\mathrm{col}}_h}
\newcommand{\tcolH}{T^{\mathrm{col}}_{H^2}}
\newcommand{\Hcomb}{H^{1\|2}}
\newcommand{\Hxor}{H^{1\oplus2}}

\newcommand{\EXP}{\textrm{EXP}}
\newcommand{\MODEXP}{\textrm{MOD{-}EXP}}
\newcommand{\ADD}{\textrm{ADD}}
\newcommand{\MULTIMODEXP}{\textrm{MULTI{-}MOD{-}EXP}}
\newcommand{\MUL}{\textrm{MUL}}
\newcommand{\MOD}{\textrm{MOD}}

\newcommand{\GG}{\mathbb{G}}
\newcommand{\ZZ}{\mathbb{Z}}

\newcommand{\bK}{\mathbf{K}}
\newcommand{\bU}{\mathbf{U}}
\newcommand{\bM}{\mathbf{M}}
\newcommand{\bC}{\mathbf{C}}

\newcommand{\rvrange}{\mathcal{R}}
\newcommand{\rspace}{\mathcal{C}}

\newcommand{\hatalpha}{\hat{\alpha}}

\newcommand{\otp}{\mathrm{OTP}}
\newcommand{\des}{\mathrm{DES}}
\newcommand{\twodes}{\mathrm{2DES}}
\newcommand{\threedes}{\mathrm{3DES}}
\newcommand{\threedestwo}{\mathrm{3DES2}}
\newcommand{\aes}{\mathrm{AES}}

\newcommand{\Img}{\mathrm{Im}}

% --- -----------------------------------------------------------------
% --- Lecture notes formatting macros
% --- -----------------------------------------------------------------

%
% The following commands set up the lecnum (lecture number)
% counter and make various numbering schemes work relative
% to the lecture number.
%
\newcounter{lecnum}
%\renewcommand{\thepage}{\thelecnum-\arabic{page}}
\renewcommand{\thesection}{\thelecnum.\arabic{section}}
\renewcommand{\theexercise}{\thelecnum.\arabic{exercise}}
\renewcommand{\theexample}{\thelecnum.\arabic{example}}
\renewcommand{\thedefinition}{\thelecnum.\arabic{definition}}
\renewcommand{\theequation}{\thelecnum.\arabic{equation}}
\renewcommand{\thefigure}{\thelecnum.\arabic{figure}}
\renewcommand{\thefact}{\thelecnum.\arabic{fact}}
\renewcommand{\thetable}{\thelecnum.\arabic{table}}


%
% The following macro is used to generate the header.
%
\newcommand{\lecture}[2]{
   %\pagestyle{myheadings}
   %\thispagestyle{plain}
   \newpage
   \setcounter{lecnum}{#1}
   \setcounter{page}{1}
   \noindent
   \begin{center}
   \framebox{
      \vbox{\vspace{2mm}
    \hbox to 6.28in { {\bf CMSC 28400 Introduction to Cryptography
                        \hfill Autumn 2020} }
       \vspace{4mm}
       \hbox to 6.28in { {\Large \hfill #2 \hfill} }
       \vspace{2mm}
       \hbox to 6.28in { {\it Instructor: David Cash} \hfill }
      \vspace{2mm}}
   }
   \end{center}
   %\markboth{Lecture #1: #2}{Lecture #1: #2}
   \vspace*{4mm}
}





% --- -----------------------------------------------------------------
% --- The document starts here.
% --- -----------------------------------------------------------------
\begin{document}
%\lecture{**LECTURE-NUMBER**}{**DATE**}{**LECTURER**}{**SCRIBE**}
\lecture{4}{Notes \#4: Block Ciphers}

Along with stream ciphers, \emph{block ciphers} are a very widely-deployed type
of algorithm used for encryption and other cryptographic purposes. These notes
cover the basic notion of a block ciphers and some interesting attacks. In the
next notes we'll learn about how block cipher are put together to build larger
algorithms, such as encryption of large files or tools for authentication.

\section{Introduction to Block Ciphers}

Let's start with the definition.
\begin{definition}
    A cipher $E:\keys\times\msgs\to\ctxts$ is called a \emph{block cipher}
    if $\msgs=\ctxts=\bits^\ell$ for some positive integer $\ell$. The integer
    $\ell$ is called the \emph{block length} or \emph{block size} of $E$.
\end{definition}
When $E$ is a block cipher, then for each $k\in\keys$, $E(k,\cdot)$ must
actually be computing a \emph{permutation} on $\bits^\ell$. In practice we will
always have $\keys=\bits^n$ for some positive integer $n$, and we say that $n$
is the \emph{key-length} of $E$.

We will sometimes want to evaluate $E^{-1}$, the decryption algorithm.  Thus
all practical block ciphers, like DES and AES that follow, support efficient
decryption.
\begin{example}
    The DES blockcipher, described in Boneh-Shoup Chapter 3 and many other
    places, is a function
    \[
        \des:\bits^{56}\times\bits^{64}\to\bits^{64}.
    \]
    The key length is $56$ and the block length is $64$. As we will see,
    both of these lengths are a major limitation.

    The replacement for DES is a family of three blockciphers, often called
    AES128, AES192, and AES256, which have a common block length of $128$
    and key lengths of $128, 192$, and $256$ respectively. The $128$
    bit version is the most commonly used. Later we'll refer back to this
    as simply \emph{the} AES blockcipher, with the notation
    \[
        \aes:\bits^{128}\times\bits^{128}\to\bits^{128}.
    \]
    We do not recall the structure of the AES block ciphers here, but you
    can read about them in Boneh-Shoup as well.
\end{example}

The construction of DES, AES, and other commonly-used block ciphers like KASUMI
is an deep and active area of cryptography. It is also highly specialized, and
unfortunately in a one-quarter introduction we can't do everything.  In
addition to Boneh-Shoup, you can also check out \emph{The Block Cipher
Companion} by Knudsen and Robshaw, which is available as a pdf from the
library: \url{https://catalog.lib.uchicago.edu/vufind/Record/8899690}.

\subsection{Security of Block Ciphers}

What does it mean for a block cipher $E:\keys\times\msgs\to\ctxts$ to be
\emph{secure}? Or, \emph{not secure}? For a stream cipher
$G:\bits^n\to\bits^\ell$ we arrived at the notion of pseudorandom generators and
distinguishing advantage, which intuitive qualified the $G$ to be used in place
of a one-time pad. For a block cipher however this definition does not fit;
It's not even clear what we should give an adversary.

In the next set of notes we will look at a formal a notion called
\emph{pseudorandom permutation distinguishing advantage}.  That definition will
probably make more sense after we look at how block ciphers are used. Just like
the pseudo-one-time-pad motivated the pseudorandom generator notion,
applications of block ciphers will drive their security requirements.

For now we'll point out some informal requirements. If we plan to encrypted
blocks with $E$, then it in almost any imaginable scenario we would want it to
be hard to recovery the blocks we are encrypting.  In a bit more detail, suppose
we are using a key $k\in\bits^n$ and sending $c_i = E(k,m_i)$ for several
blocks $m_1,m_2,\ldots\in\bits^\ell$. If an adversary captures
$c_1,c_2,\ldots$, we might ask that it be hard to recover $m_1,m_2,\ldots$.
This is called a \emph{known-ciphertext message recovery attack}, so called
because the adversary knows the ciphertexts and is targeting the input message
blocks.

An even more basic requirement is that the \emph{key $k$} be difficult to
recover from the blocks $c_1,c_2,\ldots$; call this \emph{known-ciphertext key
recovery attack}. If an adversary can recover $k$ then it can certainly recover
$m_1,m_2,\ldots$, but it is possible in principle that it finds a shortcut to
recover the message blocks without recovering the $k$; This means security
against known-ciphertext \emph{key}-recovery attacks is \emph{weaker} than
security against known-ciphertext \emph{message} recovery attacks. 

We could go on like this for a while, thinking about different combinations of
what an adversary is given and what it is trying to find.  We'll stop with one
more example that will drive a case-study enlightening some interesting issues
in the next section.

It turns out that very often an adversary learn not only some $c_1,c_2,\ldots$,
but also the input message blocks $m_1,m_2,\ldots$. Perhaps it doesn't learn
every single input message block (otherwise nothing is really hidden), but
in practice adversaries can often obtain some blocks like this. For instance,
if a web server is encrypting web pages, then adversary will very often know
that particular input blocks correspond to protocol boiler-plate text; In
that case it can match up some blocks $c_i$ with blocks $m_i$ that it knows.

This called a \emph{known plaintext attack}. Abstractly, we'll assume an
adversary is given some ``examples'' $(m_1,c_1),(m_2,c_2),\ldots$ where
$c_1=E(k,m_1), c_2=E(k,m_2),\ldots$, all computed under the same unknown key
$k$. As a natural goal for the adversary is to recover $k$, so that it can
decrypt future blocks as they're sent. This is a \emph{known-plaintext
key-recovery attack}, which is the subject of the next two sections.

\section{Exhaustive Key Search}

Consider the setting for a known-plaintext attack that has captured some
message/ciphertext pairs $(m_1,c_1),(m_2,c_2),\ldots$ where $c_i=E(k,m_i)$. A
simple strategy to find $k$ is simply try every possible key and see which one
``works''. More formally, it could work as follows:

\begin{tabbing}123\=123\=123\=\kill
    \> \textsc{Input}: Examples $(m_1,c_1),(m_2,c_2),\ldots$, where $c_i=E(k,m_i)$\\
    \> \textsc{Output}: A key $\hat{k}\in\keys$\\
    \> For $\hat{k}\in\keys$:\\
    \> \> If $E(\hat{k},m_i)=c_i$ for all $i$: Output $\hat{k}$
\end{tabbing}

There are two primary properties to consider: Correctness and runtime. The
algorithm will always terminate and output \emph{some} $\hat{k}\in\keys$,
because it will at least stop when it gets the correct key. However it may in
principle output another key that also maps all of the $m_i$ to the respective
$c_i$. In practice this basically never happens once more than a few examples
are happened; We will ignore this possibility and assume the algorithm is
correct for our purposes.

For run time, we are mostly concerned with the number of iterations of the
``For'' loop, which is $|\keys|=2^n$ for a block cipher with $n$-bit keys.  The
DES block cipher only has $2^{56}$ keys, which is not very large by modern
standards, and today even a modestly-equipped adversary can execute this
algorithm quickly (hours or minutes depending on their computer). Against AES,
the runtime is $2^{128}$, which is generally considered totally infeasible.

\section{Double Encryption}

DES was retired because of its small key length and block length, but many
systems and protocols were deployed with DES baked into them (ATMs are a famous
example). Instead of completely rebuilding the systems that used DES to convert
them to a block cipher with larger parameters like AES, many opted to use
\emph{triple encryption} where they ran DES three times instead of two. 

In this section we investigate their interesting decision to encrypt three
times rather than two. Here are the options considered, which we denote
$\twodes$ and $\threedes$. ``Double $\des$'' has the form
\[
    \twodes : \bits^{112}\times\bits^{64}\to\bits^{64}.
\]
On input $k\in\bits^{112}$ and $m\in\bits^{64}$, $\twodes$ will divide
the key into two $56$ bit keys $k_1$ and $k_2$ and outputs
\[
    \twodes(k_1\|k_2, m) = \des(k_2,\des(k_1,m)).
\]
You can check that $\twodes$ satisfies the definition of a block cipher.
The key length of $\twodes$ is $112$, and an exhaustive key search would
run on the order of $2^{112}$ time, which is likely infeasible. It would
appear that $\twodes$ is strong enough for practice. However, due to
the interesting ``meet in the middle'' attack that we look at next, $\twodes$
is not used; Instead three-key versions are used, like
\[
    \threedes : \bits^{168}\times\bits^{64}\to\bits^{64}
\]
which maps
\[
    \threedes(k_1\|k_2\|k_3, m) = \des(k_3,\des(k_2,\des(k_1,m)))
\]
where $k_1,k_2,k_3$ are $56$ bits each. In order to save on key storage,
sometimes a variant called $\threedestwo$ is used, where
\[
    \threedestwo : \bits^{112}\times\bits^{64}\to\bits^{64}
\]
and
\[
    \threedestwo(k_1\|k_2, m) = \des(k_1,\des^{-1}(k_2,\des(k_1,m))).
\]
This version seems to avoid known attacks. Note the middle application of
$\des$ is actually a decryption operation; This allows for an easy fallback
mode, where if one sets $k_1=k_2$ then $\threedestwo$ will collapse back to
``single'' $\des$.

\subsection{Meet-in-the-Middle Attack on Double Encryption}

We now give the attack that breaks $\twodes$ with about $2^{56}$ computation.
It is a known-plaintext key recovery attack, that works well with only three or
more known-plaintext blocks. To some approximation, it shows that that 
$\twodes$ is no better than single $\des$! We are looking into this because
it is an elegant method, and also because it highlights the subtleties of
security.

The attack is called \emph{meet-in-the-middle}, and we give it shortly.  This
algorithm crucially uses a \emph{hash table} with some assumed properties; We
won't get into exactly how the table is implemented, because it's pretty
standard and would involve quite a bit of detail that is not our focus.
We will just assume the table $H$ gives us the following capabilities:
\begin{itemize}
    \item Given any bit strings $x$ and $v$, we can associate $v$ with
        $x$ in constant time. We write this as
        $H[x] \gets v$.
    \item Given a bit string $x$, we can look up the string associated with
        $x$, or detect that none exists, in constant time.  We write $H[x]$ for
        the string associated with $x$; if there is none, we express this by
        saying $H[x] = \bot$.
\end{itemize}
Intuitively, $H$ is like an array, except we can use strings as indexes rather
than numbers in some range. If you've used Python dictionaries, they provide
essentially this interface.

Now for the meet-in-the-middle attack:
\begin{tabbing}123\=123\=123\=\kill
    \> \textsc{Input}: Examples $(m_1,c_1),(m_2,c_2),\ldots,(m_t,c_t)$ where
    $c_i=\twodes(k_1\|k_2,m_i)$\\
    \> \textsc{Output}: A key $\hat{k}_1\|\hat{k}_2\in\bits^{112}$\\
    \> Initialize a hash table $H$\\
    \> For $\hat{k}_1\in\bits^{56}$:\\
    \> \> $x \gets \des(\hat{k}_1,m_1)\concat \cdots \concat \des(\hat{k}_1,m_t)$\\
    \> \> $H[x] \gets \hat{k}_1$\\
    \> For $\hat{k}_2\in\bits^{56}$:\\
    \> \> $x' \gets \des^{-1}(\hat{k}_2,c_1)\concat 
           \cdots\cdots \des^{-1}(\hat{k}_2,c_t)$\\
    \> \> If $H[x'] \neq \bot$:\\
    \> \> \> $k'_1 \gets H[x']$; $k'_2 \gets \hat{k}_2$\\
    \> \> \> Output $k'_1\concat k'_2$
\end{tabbing}
What's going on in this algorithm? It is based on the following observation:
For the correct key $k_1\|k_2$, it holds that
\[
    (\des(k_1,m_1),\ldots,\des(k_1,m_t)) = 
    (\des^{-1}(k_2,c_1),\ldots,\des^{-1}(k_2,c_t)).
\]
So we can compute every possible value on each side of the equation and then
search for the collision (the ``meeting in the middle''). The trick is that
we are reusing our computational effort here. Instead of trying each
combination $k_1$ and $k_2$ together, we are able to reuse a single
evaluation with a guess for $k_1$ for \emph{every} guess of $k_2$.

The attack runs in time about $2\cdot 2^{56}=2^{57}$, which is approximately
equal to the effort required to break single $\des$.

\begin{exercise}\label{ex:examples}
    Let's estimate how the value of $t$ affects the probability that
    we get the correct key. The probability can't go down as $t$
    gets larger, but larger $t$ result in more computational effort,
    so we prefer to use some minimal value.

    Actually computing the correct value of $t$ would depend on the
    inner workings of $\des$. Instead we'll do it heuristically.
    Assume that all of the entries of $H$ are \emph{uniformly
    random and independent strings}, except for the entries that
    correspond to $k_1$ and $k_2$, which are equal. Using the union bound
    (see the probability notes), give a simple upper bound on the
    probability that the algorithm would output an incorrect key pair
    under this heuristic.
\end{exercise}

\begin{exercise}
    Adapt the meet-in-the-middle attack to $\threedes$. How long does
    your attack run, and how much memory does it consume? Repeat the
    Exercise~\ref{ex:examples} to estimate the number of examples required
    to get a good bound on the probability your algorithm will output
    the wrong key.
\end{exercise}

\begin{exercise}
    Consider the block cipher $E:\bits^{112}\times\bits^{64}\to\bits^{64}$
    defined as
    \[
        E(k_1\|k_2,m) = \des(k_1,m)\oplus k_2,
    \]
    where $k_1,k_2\in\bits^{56}$. Find a known-plaintext key
    recovery against $E$ that is not much more expensive than attacking
    plain $\des$. Repeat exercise~\ref{ex:examples}, but with your new
    attack.
\end{exercise}

\begin{exercise}
    Consider the block cipher $E:\bits^{168}\times\bits^{64}\to\bits^{64}$
    defined as
    \[
        E(k_1\|k_2\|k_3,m) = \des(k_1,m\oplus k_3)\oplus k_2,
    \]
    where $k_1\in\bits^{56}$ and $k_2,k_3\in\bits^{64}$. Find a known-plaintext
    key recovery against $E$ that takes about $2^{56+64}$ time.  Repeat
    exercise~\ref{ex:examples}, but with your new attack.
\end{exercise}



\end{document}

