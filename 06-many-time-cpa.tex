\documentclass[11pt]{article}
\usepackage{fullpage}
\usepackage{amsthm}

\usepackage{amsthm,amsmath,amsfonts,amssymb,amstext,enumitem}
\usepackage{latexsym,ifthen,url,rotating,graphicx}
\usepackage{listings}
\usepackage{tikz}
\usetikzlibrary{arrows,shapes,positioning,fit}
\usepackage{graphicx}
\usepackage[font=small,labelfont=bf]{caption}



% --- -----------------------------------------------------------------
% --- Document-specific definitions.
% --- -----------------------------------------------------------------
\lstset{
    columns=fixed,
    literate={—}{{---}}1 {…}{{...}}1
}

\newcommand{\todo}[1]{{\color{red}[TODO:{#1}]}}

\newtheorem{problem}{Problem}
\newtheorem{corollary}{Corollary}
\newtheorem{fact}{Fact}
\newtheorem{exercise}{Exercise}
\newtheorem{theorem}{Theorem}
\newtheorem{definition}{Definition}
\newtheorem{notation}{Notation}
\newtheorem{lemma}{Lemma}
\newtheorem{claim}{Claim}
\newtheorem{example}{Example}

\newcommand{\getsr}
  {{\:\stackrel{\raisebox{-2pt}{${\scriptscriptstyle \hspace{0.2em}\$}$}}
   {\leftarrow}\:}}
\newcommand{\points}[1]{\textbf{({#1} pts)}}

\newcommand{\fn}{\footnotesize}
\newcommand{\Colon}{\ : \ }
\newcommand{\st}{\mathsf{state}}
\newcommand{\msgs}{\mathcal{M}}
\newcommand{\ctxts}{\mathcal{C}}
\newcommand{\keys}{\mathcal{K}}
\newcommand{\rands}{\mathcal{R}}
\newcommand{\states}{\mathcal{S}}
\newcommand{\kg}{\mathcal{K}}
\newcommand{\Enc}{\mathsf{Enc}}
\newcommand{\Dec}{\mathsf{Dec}}
\newcommand{\MAC}{\mathrm{MAC}}
\newcommand{\RMAC}{\mathrm{RMAC}}

\newcommand{\pk}{pk}
\newcommand{\sk}{sk}

\newcommand{\calD}{\mathcal{D}}
\newcommand{\calA}{\mathcal{A}}
\newcommand{\calB}{\mathcal{B}}
\newcommand{\AES}{\mathsf{AES}}

\newcommand{\algorithm}[1]{\textbf{Alg} {#1}}

\newcommand{\calO}{\mathcal{O}}

\newcommand{\dlog}{\mathrm{dlog}}

\newcommand{\Adv}{\mathbf{Adv}}
\newcommand{\AdvPRF}[2]{\Adv^{\mathrm{prf}}_{#1}({#2})}
\newcommand{\AdvPRG}[2]{\Adv^{\mathrm{prg}}_{#1}({#2})}
\newcommand{\AdvCPA}[2]{\Adv^{\mathrm{cpa}}_{#1}({#2})}
\newcommand{\AdvCCA}[2]{\Adv^{\mathrm{cca}}_{#1}({#2})}
\newcommand{\AdvKR}[2]{\Adv^{\mathrm{kr}}_{#1}({#2})}
\newcommand{\AdvCKR}[2]{\Adv^{\mathrm{ckr}}_{#1}({#2})}
\newcommand{\AdvRMR}[2]{\Adv^{\mathrm{rmr}}_{#1}({#2})}
\newcommand{\AdvCR}[2]{\Adv^{\mathrm{cr}}_{#1}({#2})}
\newcommand{\AdvUFCMA}[2]{\Adv^{\textrm{uf{-}cma}}_{#1}({#2})}
\newcommand{\AdvDL}[2]{\Adv^{\mathrm{dl}}_{#1}({#2})}

\newcommand{\Exp}{\mathbf{Exp}}
\newcommand{\ExpOW}[1]{\Exp^{\mathrm{ow}}({#1})}
\newcommand{\ExpCKR}[2]{\Exp^{\mathrm{ckr}}_{#1}({#2})}
\newcommand{\ExpRMR}[2]{\Exp^{\mathrm{rmr}}_{#1}({#2})}

\newcommand{\concat}{{\,\|\,}}
\newcommand{\xor}{\oplus}
\newcommand{\bits}{\{0,1\}}

\newcommand{\tcolh}{T^{\mathrm{col}}_h}
\newcommand{\tcolH}{T^{\mathrm{col}}_{H^2}}
\newcommand{\Hcomb}{H^{1\|2}}
\newcommand{\Hxor}{H^{1\oplus2}}

\newcommand{\EXP}{\textrm{EXP}}
\newcommand{\MODEXP}{\textrm{MOD{-}EXP}}
\newcommand{\ADD}{\textrm{ADD}}
\newcommand{\MULTIMODEXP}{\textrm{MULTI{-}MOD{-}EXP}}
\newcommand{\MUL}{\textrm{MUL}}
\newcommand{\MOD}{\textrm{MOD}}

\newcommand{\GG}{\mathbb{G}}
\newcommand{\ZZ}{\mathbb{Z}}

\newcommand{\bK}{\mathbf{K}}
\newcommand{\bU}{\mathbf{U}}
\newcommand{\bM}{\mathbf{M}}
\newcommand{\bC}{\mathbf{C}}

\newcommand{\rvrange}{\mathcal{R}}
\newcommand{\rspace}{\mathcal{C}}

\newcommand{\hatalpha}{\hat{\alpha}}
\newcommand{\hatb}{\hat{b}}

\newcommand{\barm}{\overline{m}}

\newcommand{\otp}{\mathrm{OTP}}
\newcommand{\des}{\mathrm{DES}}
\newcommand{\twodes}{\mathrm{2DES}}
\newcommand{\threedes}{\mathrm{3DES}}
\newcommand{\threedestwo}{\mathrm{3DES2}}
\newcommand{\aes}{\mathrm{AES}}
\newcommand{\pad}{\mathsf{pad}}
\newcommand{\unpad}{\mathsf{unpad}}


\newcommand{\Img}{\mathrm{Im}}

\newcommand{\Expt}{\mathbf{Expt}}
\newcommand{\ExptOTCPA}{\mathbf{Expt}^{\mathrm{ot\mbox{-}cpa}}}
\newcommand{\ExptOTCPAone}{\mathbf{Expt}^{\mathrm{ot\mbox{-}cpa\mbox{-}1}}}
\newcommand{\ExptOTCPAzero}{\mathbf{Expt}^{\mathrm{ot\mbox{-}cpa\mbox{-}0}}}
\newcommand{\AdvOTCPA}[2]{\Adv^{\mathrm{ot\mbox{-}cpa}}_{#1}({#2})}
\newcommand{\ExptCPAone}{\mathbf{Expt}^{\mathrm{cpa\mbox{-}1}}}
\newcommand{\ExptCPAzero}{\mathbf{Expt}^{\mathrm{cpa\mbox{-}0}}}

\newcommand{\ExptCPA}{\mathbf{Expt}^{\mathrm{cpa}}}
\newcommand{\ExptCCA}{\mathbf{Expt}^{\mathrm{cca}}}
\newcommand{\LR}{\mathrm{LR}}



\newcommand{\Piotp}{\Pi_\mathrm{otp}}
\newcommand{\Encotp}{\Enc_\mathrm{otp}}
\newcommand{\Decotp}{\Dec_\mathrm{otp}}
\newcommand{\bhat}{\hat{b}}
% --- -----------------------------------------------------------------
% --- Lecture notes formatting macros
% --- -----------------------------------------------------------------

%
% The following commands set up the lecnum (lecture number)
% counter and make various numbering schemes work relative
% to the lecture number.
%
\newcounter{lecnum}
%\renewcommand{\thepage}{\thelecnum-\arabic{page}}
\renewcommand{\thesection}{\thelecnum.\arabic{section}}
\renewcommand{\theexercise}{\thelecnum.\arabic{exercise}}
\renewcommand{\theexample}{\thelecnum.\arabic{example}}
\renewcommand{\thedefinition}{\thelecnum.\arabic{definition}}
\renewcommand{\theequation}{\thelecnum.\arabic{equation}}
\renewcommand{\thefigure}{\thelecnum.\arabic{figure}}
\renewcommand{\thefact}{\thelecnum.\arabic{fact}}
\renewcommand{\thetable}{\thelecnum.\arabic{table}}


%
% The following macro is used to generate the header.
%
\newcommand{\lecture}[2]{
   %\pagestyle{myheadings}
   %\thispagestyle{plain}
   \newpage
   \setcounter{lecnum}{#1}
   \setcounter{page}{1}
   \noindent
   \begin{center}
   \framebox{
      \vbox{\vspace{2mm}
    \hbox to 6.28in { {\bf CMSC 28400 Introduction to Cryptography
                        \hfill Autumn 2020} }
       \vspace{4mm}
       \hbox to 6.28in { {\Large \hfill #2 \hfill} }
       \vspace{2mm}
       \hbox to 6.28in { {\it Instructor: David Cash} \hfill }
      \vspace{2mm}}
   }
   \end{center}
   %\markboth{Lecture #1: #2}{Lecture #1: #2}
   \vspace*{4mm}
}





% --- -----------------------------------------------------------------
% --- The document starts here.
% --- -----------------------------------------------------------------
\begin{document}
%\lecture{**LECTURE-NUMBER**}{**DATE**}{**LECTURER**}{**SCRIBE**}
\lecture{6}{Notes \#6: Randomized Encryption and Chosen-Plaintext Security}

Recall that at the beginning of Notes \#3 we identified three crucial
weaknesses of the OTP: Long keys, one-time security, and malleability.
PRGs allowed us to address the first issue, and in these notes we begin
with the second.


\section{Many-Time Security and Randomized Encryption}

The OTP and psuedo-OTP are \emph{ciphers}, which means they are functions
satisfying a certain property we defined in the very first set of notes.
Implicit in the definition of a cipher is that encryption is ``deterministic,''
meaning that if one encrypts the same message with the same key multiple times,
then the same ciphertext will be emitted each time. A closer look at these
ciphers reveals that their security behavior when encrypting multiple messages
is even more concerning. For instance, suppose we use the OTP to encrypted a
message $m_1$, producing a ciphertext $c_1$. Later, say we want to encrypt
message $m_2$ that is exactly $m_1$, but with its first bit flipped.  Using the
same key, we would produce a ciphertext $c_2$ that is equal to $c_1$, except
with its first bit flipped! 

Should this property concern us, as cryptographers? Security always depends on
the application, and the OTP has its place. But here is an example of the sort
of thinking that suggests we should consider stronger security: Suppose we
encrypt a super-secret number $m$ using the OTP, which is the number of widgets
sold so far this year. Each day we add to $m$ the number of sales, and encrypt
the number again, producing a new ciphertext. Now consider the point of view of
an adversary: It will see a sequence of ciphertexts, and be able to detect
which bits have been changed each time. This will tell the adversary a lot
about how our day-to-day sales, even though everything is encrypted, and each
ciphertext is (in some sense) ``unbreakable'' on its own.

This issue played a role in some classical ciphers, like the homophonic cipher
you broke in Project 1. Recall that this wasn't really a cipher, since the
encryption process was not really a function but a recipe of sorts which could
produce different ciphertexts when run on the same input twice. Another example
is the famous Enigma cipher used in the mid 20th century, notably by the
Germans in World War II. That cipher explicitly instructed senders to pick a
random ``message key'' to \emph{randomize} ciphertexts.

Basically all commonly-used modern encryption is designed to avoid these
problems when encrypting multiple messages with the same key. (The ones that
aren't are explicitly marked as good for one-time use only.) Because of the
deterministic nature of a cipher $E$, we need a new definition that allows for
randomization. The following allows for this, and is called an \emph{encryption
scheme}. The name is just a convention, but it is pretty standard. Intuitively,
a randomized encryption is like a cipher, except the encryption direction takes
an extra input that can be used to randomized the process. Decryption still
have to recover the message, but doesn't to recover that extra input.
\begin{definition}
    A pair of functions $\Pi=(\Enc,\Dec)$,
    \begin{align*}
        \Enc  : \keys\times\msgs\times\rands  \to  \ctxts 
    \end{align*}
    and
    \begin{align*}
        \Dec : \keys\times\ctxts \to \msgs,  
    \end{align*}
    is called a \emph{randomized encryption scheme with key-space $\keys$,
    message-space $\msgs$, randomness-space $\rands$, and ciphertext-space
    $\ctxts$} if for every $k\in\keys$, $m\in\msgs$, $r\in\rands$, and
    $c\in\ctxts$
    \begin{align*}
        \Enc(k,m,r)=c \quad \implies \quad \Dec(k,c)=m.
    \end{align*}
\end{definition}
The intended usage is for someone to pick $r\in\rands$ and then run
$c\gets\Enc(k,m,r)$. An interesting point is that decryption must work with $k$
and $c$ \emph{only}. In particular, the value $r$ chosen by the sender is not
assumed to be available for decryption unless it is communicated in $c$; Of
course one can always put $r$ there, but doing so may have implications for
security.

The OTP and pseudo-OTP do not really fit this definition (if $\rands$ is a
trivial set or something they can be formally viewed this way, but that is not
the spirit of the definition). Intuitively, it's not even clear syntactically
how to mix in randomness into these ciphers. At the end of these notes we'll
begin looking at how randomness might be used, but with a block cipher.

\section{Chosen-Plaintext Security of Randomized Encryption}

We next formalize a security goal for randomized encryption schemes. The idea
is to require that an adversary cannot learn anything non-trivial about
(several) encrypted messages. It should not, in particular, be able to detect
if the same message was encrypted multiple times or not. We'd also like our
encryption scheme to securely encrypt any type of messages selected by an
arbitrary underlying protocol (e.g. HTTP), and not rely on messages avoiding
any pathologies.

More interestingly is that we want our encryption to remain secure \emph{even
when an adversary can influence what we encrypt}. Here is a modern example
where an adversary has this type of power: Suppose we are encrypting traffic
between a front-end web server and a backend database holding user account
information. Anyone on the internet can create a free account, and upon account
creation, the username $u$ and password $p$ are encrypted by the webserver and
sent to the database. Now suppose the link between these machines is visible to
an adversary. This adversary\footnote{This adversary may really be multiple
coordinating people or machines, but we consider it to \emph{one} adversary.}
can create accounts just like anyone else! Thus it can choose $u$ and $p$ as it
likes, and then observe the resulting ciphertext. Moreover, it can do this many
times, perhaps thousands.

More subtle versions of this problem can arise as well. Sometimes adversaries
will know we're encrypting information from a network protocol. By interfering
with our connection, it may be possible for the adversary to induce us to
encrypt predictable error messages.

With that motivation, we give a definition that should hold up those types
of attacks: The solution is to make a definition where the adversary gets
to pick the messages, sort of like in one-time CPA, but now for many
messages. Formally, this definition uses the concept of an \emph{oracle}.
You can think about oracles intuitively as ``subroutines'' that an
algorithm can call. When $\calA$ is an oracle algorithm and $O_1$ is a
function, we write $\calA^{O_1}$ for $\calA$ connected to the oracle
(subroutine) $O_1$. If $O_2$ is another oracle, we write $\calA^{O_2}$ for
$\calA$ connected to $O_2$, and so on.  A key point in this formalism is that
$\calA$ can only observe the input/output behavior of its oracle, and \emph{not
the code implementing the oracle}.  So if $O_1$ and $O_2$ are the same
function, then $\calA^{O_1}$ and $\calA^{O_2}$ will behave exactly the same. 

Now for the definition.
\begin{definition}
    Let $\Pi = (\Enc,\Dec)$ be a randomized encryption scheme with key-space
    $\keys$, message-space $\msgs\subseteq\bits^*$, randomness-space $\rands$,
    and ciphertext-space $\ctxts$.  We assume that the message-space $\msgs$ is
    a set of bit strings, i.e.  $\msgs\subseteq\bits^*$.  Let $\calA$ be an
    algorithm. Define algorithm $\ExptCPA_\Pi(\calA)$ as
    \begin{center}
    \begin{tabular}{c}
        \begin{minipage}{2in}\begin{tabbing}
            123\=123\=\kill
            \underline{\algorithm{$\ExptCPA_\Pi(\calA)$}} \\[2pt]
            \fn01 \> Pick $k\getsr \keys, b\getsr \bits$\\
            \fn02 \> Run $\calA^{\LR_{k,b}(\cdot,\cdot)}$, where the oracle
            is given below. Eventually $\calA$ halts with output $\hatb$\\
            \fn03\> If $\hatb = b$: Output 1\\
            \fn06 \> Else: Output 0\\
            \\
            \underline{Oracle $\LR_{k,b}(m_0,m_1)$} \\
            \> If $m_0,m_1$ are not the same length: Return $\bot$\\
            \> Pick $r \getsr \rands$\\
            \> Compute $c \gets \Enc(k,m_b,r)$\\
            \> Return $c$.
        \end{tabbing}\end{minipage}
    \end{tabular}
    \end{center}
    Define the \emph{CPA advantage of $\calA$ against $\Pi$} as
    \[
        \AdvCPA{\Pi}{\calA} =
        \left|\Pr[\ExptCPA_\Pi(\calA) = 1] - \frac{1}{2}\right|.
    \]
\end{definition}
This definition has a lot in common with the one-time CPA definition. The
adversary is picking left/right messages $m_0,m_1$ like in that definition,
except now it is given as many shots as it likes (subject to any limitations on
its runtime, since each query costs at least one basic operation). Note that
each query is answered by picking a fresh, independent value $r$ to be used
for encryption. Also, every call to the oracle uses the same values of
$k$ and $b$. That is, the same key is used, and whether the left or right
message is encrypted is determined across all queries, not per-query (so either
\emph{every} query is answered by encrypting the left message, or \emph{every}
query is answered by encrypting the right message).

In practice we are concerned with adversaries $\calA$ with a similar level of
resources to those we considered before. For example, we might hope for
security against an adversary running time $2^{128}$ and issuing $2^{128}$
queries, achieving advantage $2^{-128}$. As before, the maximum advantage
is $1/2$, and anything remotely large (like $2^{-50}$) is considered to be
a complete break.

\bigskip

The next claim shows that deterministic encryption schemes cannot have
good CPA security.
\begin{claim}
    Suppose $\Pi=(\Enc,\Dec)$ is an encryption scheme with
    $\keys,\msgs,\rands$, where $\msgs\subseteq\bits^*$ contains at least two 
    equal-length messages. Assume
    $\Enc$ is deterministic (i.e. it does not use its input from $\rands$).
    Then there exits an adversary $\calA$ making two queries such that
    \[
        \AdvCPA{\Pi}{\calA} = 1/2.
    \]
\end{claim}
\begin{proof}
    Let $m,m'\in\msgs$ be distinct messages of the same length. The adversary
    $\calA$ works as follows: Given access to an oracle $\calO(\cdot,\cdot)$,
    do the following:

        \begin{tabbing}
            123\=123\=\kill
            \> Query $\calO(m,m)$. Call the response $c_1$\\
            \> Query $\calO(m,m')$. Call the response $c_2$\\
            \> If $c_1 = c_2$ output $0$, Else output $1$.
        \end{tabbing}

    We claim that $\Pr[\ExptCPA_\Pi(\calA) = 1]$. This is equivalent
    to saying that $\calA$ always outputs the correct bit $\hatb=b$.
    This follows by the assumption that $\Enc$ is deterministic:
    if $b=0$, then $c_1$ and $c_2$ are both $\Enc(k,m)$ and $\calA$
    outputs $\hatb=0$. If $b=1$ then $c_1 =\Enc(k,m)$ and $c_2 =\Enc(k,m')$.
    But then $c_1$ and $c_2$ cannot be equal, because decryption
    must be correct. Thus $\calA$ will output $\hatb=1$ in this case,
    completely the proof.
\end{proof}

\begin{exercise}
    Check the above proof and pinpoint where we used the fact that
    $m$ and $m'$ have the same length.
\end{exercise}

Notice how the adversary above was described: It's defined to work with
\emph{any} oracle $\calO$ that fits the correct syntax (i.e. accepting two
messages). We have to describe $\calA$ in this way because it cannot ``see''
what oracle its interacting with. But when it comes time to analyzing $\calA$,
we actually instantiate $\calO$ with a particular function and calculate
probabilities.

\begin{exercise}
    Let $\Pi$ be the OTP cipher, viewed as an encryption scheme with
    deterministic encryption scheme. Describe a simple adversary $\calA$
    that never inputs the same message to its oracle more than once and yet
    $\AdvCPA{\Pi}{\calA} = 1/2$. Note the above adversary inputs $m$
    more than once (in fact three times) to its oracle, so that adversary
    does not count.
\end{exercise}

\begin{exercise}
    Define $\Enc:\keys\times\msgs\times\rands\to\ctxts$, where 
    $\keys=\msgs=\rands=\bits^\ell$ and $\ctxts = \bits^{2\ell}$, 
    as
    \[
        \Enc(k,m,r) = (r\oplus k, m\oplus r).
    \]
    \begin{itemize}

        \item Give an algorithm $\Dec$ that makes $\Pi = (\Enc,\Dec)$ into a
            randomized encryption scheme (i.e. so that $\Dec$ correctly
            recovers the message).

        \item Give a simple adversary $\calA$ such that $\AdvCPA{\Pi}{\calA} =
            1/2$.

    \end{itemize}

\end{exercise}




\end{document}

