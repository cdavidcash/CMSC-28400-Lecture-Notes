\documentclass[11pt]{article}
\usepackage{fullpage}
\usepackage{amsthm}

\usepackage{amsthm,amsmath,amsfonts,amssymb,amstext,enumitem}
\usepackage{latexsym,ifthen,url,rotating,graphicx}
\usepackage{listings}
\usepackage{tikz}
\usetikzlibrary{arrows,shapes,positioning,fit}
\usepackage{graphicx}
\usepackage[font=small,labelfont=bf]{caption}



% --- -----------------------------------------------------------------
% --- Document-specific definitions.
% --- -----------------------------------------------------------------
\lstset{
    columns=fixed,
    literate={—}{{---}}1 {…}{{...}}1
}

\newcommand{\todo}[1]{{\color{red}[TODO:{#1}]}}

\newtheorem{problem}{Problem}
\newtheorem{corollary}{Corollary}
\newtheorem{fact}{Fact}
\newtheorem{exercise}{Exercise}
\newtheorem{theorem}{Theorem}
\newtheorem{definition}{Definition}
\newtheorem{notation}{Notation}
\newtheorem{lemma}{Lemma}
\newtheorem{example}{Example}

\newcommand{\getsr}
  {{\:\stackrel{\raisebox{-2pt}{${\scriptscriptstyle \hspace{0.2em}\$}$}}
   {\leftarrow}\:}}
\newcommand{\points}[1]{\textbf{({#1} pts)}}

\newcommand{\fn}{\footnotesize}
\newcommand{\Colon}{\ : \ }
\newcommand{\st}{\mathsf{state}}
\newcommand{\msgs}{\mathcal{M}}
\newcommand{\ctxts}{\mathcal{C}}
\newcommand{\keys}{\mathcal{K}}
\newcommand{\rands}{\mathcal{R}}
\newcommand{\states}{\mathcal{S}}
\newcommand{\kg}{\mathcal{K}}
\newcommand{\Enc}{\mathsf{Enc}}
\newcommand{\Dec}{\mathsf{Dec}}
\newcommand{\MAC}{\mathrm{MAC}}
\newcommand{\RMAC}{\mathrm{RMAC}}

\newcommand{\pk}{pk}
\newcommand{\sk}{sk}

\newcommand{\calD}{\mathcal{D}}
\newcommand{\calA}{\mathcal{A}}
\newcommand{\calB}{\mathcal{B}}
\newcommand{\AES}{\mathsf{AES}}

\newcommand{\algorithm}[1]{\textbf{Alg} {#1}}

\newcommand{\calO}{\mathcal{O}}

\newcommand{\dlog}{\mathrm{dlog}}

\newcommand{\Adv}{\mathbf{Adv}}
\newcommand{\AdvPRF}[2]{\Adv^{\mathrm{prf}}_{#1}({#2})}
\newcommand{\AdvPRG}[2]{\Adv^{\mathrm{prg}}_{#1}({#2})}
\newcommand{\AdvCPA}[2]{\Adv^{\mathrm{cpa}}_{#1}({#2})}
\newcommand{\AdvCCA}[2]{\Adv^{\mathrm{ind{-}cca}}_{#1}({#2})}
\newcommand{\AdvKR}[2]{\Adv^{\mathrm{kr}}_{#1}({#2})}
\newcommand{\AdvCKR}[2]{\Adv^{\mathrm{ckr}}_{#1}({#2})}
\newcommand{\AdvRMR}[2]{\Adv^{\mathrm{rmr}}_{#1}({#2})}
\newcommand{\AdvCR}[2]{\Adv^{\mathrm{cr}}_{#1}({#2})}
\newcommand{\AdvUFCMA}[2]{\Adv^{\textrm{uf{-}cma}}_{#1}({#2})}
\newcommand{\AdvDL}[2]{\Adv^{\mathrm{dl}}_{#1}({#2})}

\newcommand{\Exp}{\mathbf{Exp}}
\newcommand{\ExpOW}[1]{\Exp^{\mathrm{ow}}({#1})}
\newcommand{\ExpCKR}[2]{\Exp^{\mathrm{ckr}}_{#1}({#2})}
\newcommand{\ExpRMR}[2]{\Exp^{\mathrm{rmr}}_{#1}({#2})}

\newcommand{\concat}{{\,\|\,}}
\newcommand{\xor}{\oplus}
\newcommand{\bits}{\{0,1\}}

\newcommand{\tcolh}{T^{\mathrm{col}}_h}
\newcommand{\tcolH}{T^{\mathrm{col}}_{H^2}}
\newcommand{\Hcomb}{H^{1\|2}}
\newcommand{\Hxor}{H^{1\oplus2}}

\newcommand{\EXP}{\textrm{EXP}}
\newcommand{\MODEXP}{\textrm{MOD{-}EXP}}
\newcommand{\ADD}{\textrm{ADD}}
\newcommand{\MULTIMODEXP}{\textrm{MULTI{-}MOD{-}EXP}}
\newcommand{\MUL}{\textrm{MUL}}
\newcommand{\MOD}{\textrm{MOD}}

\newcommand{\GG}{\mathbb{G}}
\newcommand{\ZZ}{\mathbb{Z}}

\newcommand{\bK}{\mathbf{K}}
\newcommand{\bof}{\mathbf{f}}
\newcommand{\bU}{\mathbf{U}}
\newcommand{\bM}{\mathbf{M}}
\newcommand{\bC}{\mathbf{C}}

\newcommand{\rvrange}{\mathcal{R}}
\newcommand{\rspace}{\mathcal{C}}

\newcommand{\hatalpha}{\hat{\alpha}}
\newcommand{\hatb}{\hat{b}}

\newcommand{\barm}{\overline{m}}

\newcommand{\otp}{\mathrm{OTP}}
\newcommand{\des}{\mathrm{DES}}
\newcommand{\twodes}{\mathrm{2DES}}
\newcommand{\threedes}{\mathrm{3DES}}
\newcommand{\threedestwo}{\mathrm{3DES2}}
\newcommand{\aes}{\mathrm{AES}}
\newcommand{\pad}{\mathsf{pad}}
\newcommand{\unpad}{\mathsf{unpad}}
\newcommand{\Func}{\mathrm{Func}}


\newcommand{\Img}{\mathrm{Im}}

\newcommand{\Expt}{\mathbf{Expt}}
\newcommand{\ExptCPA}{\mathbf{Expt}^{\mathrm{cpa}}}
\newcommand{\ExptCCA}{\mathbf{Expt}^{\mathrm{cca}}}
\newcommand{\ExptOTCPA}{\mathbf{Expt}^{\mathrm{1\mbox{-}cpa}}}
\newcommand{\ExptOTCPAone}{\mathbf{Expt}^{\mathrm{1\mbox{-}cpa\mbox{-}1}}}
\newcommand{\ExptOTCPAzero}{\mathbf{Expt}^{\mathrm{1\mbox{-}cpa\mbox{-}0}}}
\newcommand{\AdvOTCPA}[2]{\Adv^{\mathrm{1\mbox{-}cpa}}_{#1}({#2})}
\newcommand{\ExptCPAone}{\mathbf{Expt}^{\mathrm{cpa\mbox{-}1}}}
\newcommand{\ExptCPAzero}{\mathbf{Expt}^{\mathrm{cpa\mbox{-}0}}}

\newcommand{\LR}{\mathrm{LR}}

\newcommand{\Piotp}{\Pi_\mathrm{otp}}
\newcommand{\Encotp}{\Enc_\mathrm{otp}}
\newcommand{\Decotp}{\Dec_\mathrm{otp}}
\newcommand{\bhat}{\hat{b}}
\newcommand{\dict}{\mathtt{dict}}
\newcommand{\Col}{\mathsf{Col}}
% --- -----------------------------------------------------------------
% --- Lecture notes formatting macros
% --- -----------------------------------------------------------------

%
% The following commands set up the lecnum (lecture number)
% counter and make various numbering schemes work relative
% to the lecture number.
%
\newcounter{lecnum}
%\renewcommand{\thepage}{\thelecnum-\arabic{page}}
\renewcommand{\thesection}{\thelecnum.\arabic{section}}
\renewcommand{\theexercise}{\thelecnum.\arabic{exercise}}
\renewcommand{\theexample}{\thelecnum.\arabic{example}}
\renewcommand{\thedefinition}{\thelecnum.\arabic{definition}}
\renewcommand{\theequation}{\thelecnum.\arabic{equation}}
\renewcommand{\thefigure}{\thelecnum.\arabic{figure}}
\renewcommand{\thefact}{\thelecnum.\arabic{fact}}
\renewcommand{\thetable}{\thelecnum.\arabic{table}}


%
% The following macro is used to generate the header.
%
\newcommand{\lecture}[2]{
   %\pagestyle{myheadings}
   %\thispagestyle{plain}
   \newpage
   \setcounter{lecnum}{#1}
   \setcounter{page}{1}
   \noindent
   \begin{center}
   \framebox{
      \vbox{\vspace{2mm}
    \hbox to 6.28in { {\bf CMSC 28400 Introduction to Cryptography
                        \hfill Autumn 2019} }
       \vspace{4mm}
       \hbox to 6.28in { {\Large \hfill #2 \hfill} }
       \vspace{2mm}
       \hbox to 6.28in { {\it Instructor: David Cash} \hfill }
      \vspace{2mm}}
   }
   \end{center}
   %\markboth{Lecture #1: #2}{Lecture #1: #2}
   \vspace*{4mm}
}





% --- -----------------------------------------------------------------
% --- The document starts here.
% --- -----------------------------------------------------------------
\begin{document}
%\lecture{**LECTURE-NUMBER**}{**DATE**}{**LECTURER**}{**SCRIBE**}
\lecture{9}{Notes \#9: Pseudorandom Functions and (Many-Time) CPA Security}

\tableofcontents

\noindent\hrulefill
\bigskip

\section{Pseudorandom Functions}

We now give an abstraction describing a main security goal of block ciphers.
In this definition we will need the concept of an \emph{oracle} for an
algorithm.  You can think about oracles intuitively as ``subroutines'' that an
algorithm can call. When $\calA$ is an oracle algorithm and $O_1$ is a
function, we write $\calA^{O_1}$ for $\calA$ connected to the oracle
(subroutine) $O_1$. If $O_2$ is another oracle, we write $\calA^{O_2}$ for
$\calA$ connected to $O_2$, and so on.  A key point in this formalism is that
$\calA$ can only observe the input/output behavior of its oracle, and \emph{not
the code implementing the oracle}.  So if $O_1$ and $O_2$ are the same
function, then $\calA^{O_1}$ and $\calA^{O_2}$ will behave exactly the same. 

The following definition also needs the concept of a \emph{random function}
$\bof:\bits^\ell\to\bits^\ell$. You can think of such a random variable
as a table listing an output for every possible input, and such that
the outputs are all uniformly random and independent. For a concrete
example with $\ell=3$, we might have $\bof$ represented by the table:
\begin{center}
\begin{tabular}{c|c}
    $x$ & $f(x)$ \\\hline
    $000$ & $100$ \\
    $001$ & $011$ \\
    $010$ & $000$ \\
    $011$ & $101$ \\
    $100$ & $011$ \\
    $100$ & $111$ \\
    $101$ & $001$ \\
    $110$ & $000$ \\
    $111$ & $110$
\end{tabular}
\end{center}
So ``picking a random function'' means filling in the right side of the table
with uniformly random and independent entries. These entries may repeat, but
the outcome of one entry does not influence the outcome of any other entry.

\begin{example}
Let $\bof$ be a random function from $\bits^{128}$ to $\bits^{128}$.
Then the following hold:
    \begin{align*}
        \Pr[\bof(0^{128}) = 0^{128}] &= \frac{1}{128}, \\
        \Pr[\bof(0^{128}) \text{ starts with $0$}] &= \frac{1}{2}, \\
        \Pr[\bof(0^{128}) = \bof(1^{128})] &= \frac{1}{128}, \\
        \Pr[\bof(0^{128}) \oplus \bof(1^{128}) = 1^{128}] &= \frac{1}{128}.
    \end{align*}
    These can be proved rigorously by the independence and uniformity of
    $\bof(0^{128})$ and $\bof(1^{128})$ (or, one could actually count
    the fraction of all possible functions for which these hold).
    
    Things are not always so simple. For instance,
    \begin{align*}
        \Pr[\bof(\bof(0^{128})) = 0^{128}] &= 
        \frac{1}{2^{128}} + (1-\frac{1}{2^{128}})\frac{1}{2^{128}},
    \end{align*}
    which can be seen by applying the law of total probability to split
    up the probability into the cases where $f(0^{128})=0^{128}$ and
     $f(0^{128})\neq0^{128}$.
\end{example}

\begin{exercise}
    Let $\Func(\bits^\ell,\bits^\ell)$ be the set of all functions from
$\bits^\ell$ to $\bits^\ell$. This set has size $2^{\ell2^\ell}$. Using the
table point of view on functions, or another method, explain this formula.
\end{exercise}

We can now state the definition, and afterwards interpret it.
\begin{definition}
    Let $E:\bits^n\times\bits^\ell\to\bits^\ell$ and $\calA$ be an adversary.
    Let $\bK$ be uniform on $\bits^n$, and let $\bof$ be a random function
    from $\bits^\ell$ to $\bits^\ell$.
    We define the \emph{pseudorandom function (PRF) distinguishing advantage
    of $\calA$ against $E$} to be
    \[
        \AdvPRF{E}{\calA} =
        \left|\Pr[\calA^{E(\bK,\cdot)}=1]-\Pr[\calA^{\bof(\cdot)}=1]\right|.
    \]
\end{definition}

An adversary $\calA$ can be judged according to its runtime, the number of
queries it issues, and its advantage. We won't distinguish much between runtime
and queries, but in practice an adversary will usually have far more
computation time than it does queries, because the queries need to be run by
the parties under attack. They won't be enabling anything like $2^{80}$ queries,
but the adversary might have that much computation.  If the advantage of every
reasonable $\calA$ is small (say $\frac{1}{2^{64}}$ or $\frac{1}{2^{128}}$,
depending on the application), we informally say that $E$ is a good PRF. 

In that case we think of $E$ as ``essentially looking like a
random function,'' and all of the properties of random functions are thus
inherited by $E$. So $E(\bK,x)$ and $E(\bK,x')$ should look like totally
random, independent strings, when $x\neq x'$, even when they differ only by one
bit. If not, then there would be an adversary with good PRF advantage.

\begin{example}
    Suppose $E:\bits^{128}\times\bits^{128}\to\bits^{128}$ satisfies
    $E(k,0^{128})=0^{128}$ for every $k\in\bits^{128}$. We show that
    $E$ is not a good PRF. Consider $\calA^\calO$ that queries $0^{128}$
    to its oracle $\calO$ and calls the response $y$. If $y=0^{128}$
    then $\calA$ outputs $1$, else $0$. Then
    \[
        \Pr[\calA^{E(\bK,\cdot)}=1]=1
    \]
    and
    \[
        \Pr[\calA^{\bof(\cdot)}=1] = \frac{1}{2^{128}},
    \]
    the latter because a random function satisfies $\bof(0^{128})=0^{128}$
    with probability $\frac{1}{2^{128}}$. Thus
    \[
        \AdvPRF{E}{\calA} = 1 - \frac{1}{2^{128}},
    \]
    which is high.
\end{example}
Note that we had to define $\calA$ with respect to a generic oracle $\calO$;
We don't get to say how $\calA$ works with $E(\bK,\cdot)$ and $\bof(\cdot)$
separately, since $\calA$ only gets to see their input/output behavior.
More complicated patterns can be found with more clever attacks.
\begin{example}
    Define $E:\bits^{128}\times\bits^{128}\to\bits^{128}$ by
    $E(k,x)=k\oplus x$.
    We show that
    $E$ is not a good PRF. Consider $\calA^\calO$ that fixes any
    $x_1\neq x_2\in\bits^{128}$, 
    queries $y_1 \gets \calO(x_1)$, and
    $y_2 \gets \calO(x_2)$. 
    If $y_1\oplus y_2 = x_1 \oplus x_2$ then $\calA$ outputs $1$, 
    else it outputs $0$.
    \[
        \Pr[\calA^{E(\bK,\cdot)}=1]=1
    \]
    and
    \[
        \Pr[\calA^{\bof(\cdot)}=1] 
        \Pr[\bof(x_1)\oplus \bof(x_2)=x_1\oplus x_2] = \frac{1}{2^{128}},
    \]
    the latter because a random function satisfies that
    equation
    with probability $\frac{1}{2^{128}}$ (because $\bof(x_1),\bof(x_2)$ are
    uniform and independent). Thus
    \[
        \AdvPRF{E}{\calA} = 1 - \frac{1}{2^{128}},
    \]
    which is high. Note that we needed two queries to break this one -- It's
    actually impossible to break it in one query!
\end{example}

Finally we address the need for oracles in the definition. It would be simpler
to, say, provide a function table (like the one we diagrammed above) to
$\calA$, and ask it if the table represents $E(\bK,\cdot)$ or $\bof$.  But this
table would gigantic for the sizes we care about: $2^{128}\cdot 128$ bits for
$\AES$. But no algorithm with realistic resources could even read its input in
that case. We resort to oracles to allow ``selective'' access to parts of
the function without paying a huge runtime.

\section{Birthday Attacks}

You may have noticed that in the definition of PRF advantage, when
the oracle is $E(\bK,\cdot)$, then the adversary will never see
a repeated output for two distinct inputs. But when the oracle
is $\bof(\cdot)$ then two distinct inputs will produce the same
output with non-zero probability. This suggests the following adversary, which
attacks $E:\bits^n\times\bits^k\to\bits^k$ and issues some number $q\leq
2^{\ell}$ queries: 
\begin{center}
    \begin{tabular}{c}
        \begin{minipage}{2in}\begin{tabbing}
            123\=123\=\kill
            \underline{Adversary $\calA^{\calO}$}\\[2pt]
            \> Initialize an empty set $S$ \\
            \> For $i=1,\ldots,q$: \\
            \> \> Let $x_i\in\bits^\ell$ be the $\ell$-bit encoding of $i$\\
            \> \> Query $y_i \gets \calO(x_i)$\\
            \> \> If $y_i \in S$: Output $1$ \\
            \> \> Add $y_i$ to $S$\\
            \> Output $0$
        \end{tabbing}\end{minipage}
    \end{tabular}
\end{center}
The set $S$ could be implemented with a hash table or another data structure
that allows for fast membership testing and fast addition.

Let's find the advantage of this adversary. First it is simple to show
that
\[
    \Pr[\calA^{E(\bK,\cdot)}=1]=0
\]
because the $y_i$ will all be distinct points, and hence will never
be in $S$ in the for loop. This is because the $y_i$ are the outputs
of a block cipher applied with the same key to distinct inputs.
On the other hand,
\[
    \Pr[\calA^{\bof(\cdot)}=1]
\]
is less obvious to calculate. We will need the following interesting
and well-known ``birthday bound''.

\begin{theorem}
    Let $N,q\geq 1$ be integers, and let $\Col(N,q)$ be the probability that
    there is a repeated value (a \emph{collision}) in $q$ independent uniform
    samples from a set of size $N$. Then
    \[
        1 - \exp{-q(q-1)/2N} \leq \Col(N,q) \leq 0.5\frac{q(q-1)}{N}.
    \]
    If $q\leq \sqrt{2N}$, then this implies
    \[
        0.3 \frac{q(q-1)}{N} \leq \Col(N,q) \leq 0.5\frac{q(q-1)}{N}.
    \]
\end{theorem}
Above $\exp(x)=e^x$ is the usual exponential function from calculus.
\begin{exercise}
    Prove the upper bound of the theorem via a union bound.
\end{exercise}
The lower bound is only slightly harder to prove; See page 273 of
\url{https://web.cs.ucdavis.edu/~rogaway/classes/227/spring05/book/main.pdf}.

What is surprising about this bound is that the probability of a collision
is approximately $q^2/N$ and not $q/N$. In particular, once $q$ is only
a little larger than $\sqrt{N}$, the probability jumps up exponentially close
$1$ and is therefore almost certain. This results in some surprising
phenomena, like the result of the following exercise
\begin{exercise}
    Use the theorem to find the minimum number $q$ of people required to have
    a $1/2$ or greater chance of two people having the same birthday. Assume
    birthdays are uniformly random and independent values amongst the
    $365$ days of the year.
\end{exercise}

Returning to our adversary, and assuming $q\leq \sqrt{2\cdot 2^\ell}$, we
have
\[
    \Pr[\calA^{\bof(\cdot)}=1] \geq 0.3 \frac{q(q-1)}{2^\ell},
\]
because the $y_i$ values in the algorithm are all uniform and independent
samples from a set of size $2^\ell$ when the oracle is a random function. 
Thus
\begin{align*}
    \AdvPRF{E}{\calA} 
    & = \left|\Pr[\calA^{E(\bK,\cdot)}=1]-\Pr[\calA^{\bof(\cdot)}=1]\right| \\
    & = \left|0-\Pr[\calA^{\bof(\cdot)}=1]\right| \\
    & = \Pr[\calA^{\bof(\cdot)}=1] \\
    & \geq  0.3 \frac{q(q-1)}{2^\ell}.
\end{align*}
Thus, for $\ell$-bit outputs, we only need $q\approx 2^{\ell/2}$, not
$q\approx 2^{\ell}$ queries, to have good PRF advantage! For AES, this
means about $2^{64}$ queries suffice, which is astronomically smaller
than $2^{128}$.

These attacks are called ``birthday attacks,'' and are avoidable if one
insists on working with a block cipher (as we do). Thus we'll aim for
a block cipher to have \emph{only} this defect, and remain a good PRF
until about $2^{\ell/2}$ queries are issued.

\section{Chosen-Plaintext Security}

We now extend the one-time CPA definition to many queries.

\begin{definition}
    Let $\Pi = (\Enc,\Dec)$ be a randomized encryption scheme with key-space
    $\keys$, message-space $\msgs$, randomness-space $\rands$, and
    ciphertext-space $\ctxts$.  We assume that the message-space $\msgs$ is a
    set of bit strings, i.e.  $\msgs\subseteq\bits^*$.  Let $\calA$ be an
    algorithm. Define algorithm $\ExptCPA_\Pi(\calA)$ as
    \begin{center}
    \begin{tabular}{c}
        \begin{minipage}{2in}\begin{tabbing}
            123\=123\=\kill
            \underline{\algorithm{$\ExptCPA_\Pi(\calA)$}} \\[2pt]
            \fn01 \> Pick $k\getsr \keys, b\getsr \bits$\\
            \fn02 \> Run $\calA^{\LR_{k,b}(\cdot,\cdot)}$, where the oracle
            is given below. Eventually $\calA$ halts with output $\hatb$\\
            \fn03\> If $\hatb = b$: Output 1\\
            \fn06 \> Else: Output 0\\
            \\
            \underline{Oracle $\LR_{k,b}(m_0,m_1)$} \\
            \> If $m_0,m_1$ are not the same length: Return $\bot$\\
            \> Pick $r \getsr \rands$\\
            \> Compute $c \gets \Enc(k,m_b,r)$\\
            \> Return $c$
        \end{tabbing}\end{minipage}
    \end{tabular}
    \end{center}
    Define the \emph{CPA advantage of $\calA$ against $\Pi$} as
    \[
        \AdvCPA{\Pi}{\calA} =
        \left|\Pr[\ExptCPA_\Pi(\calA) = 1] - \frac{1}{2}\right|
    \]

    The definition for the stateful version of the scheme
    is the same, except the oracle works as follows:
    \begin{center}
        \begin{tabular}{c}
            \begin{minipage}{2in}\begin{tabbing}
                123\=123\=\kill
                \underline{Oracle $\LR_{k,b}(m_0,m_1)$} \\
                \> If $m_0,m_1$ are not the same length: Return $\bot$\\
                \> Compute $(c,s') \gets \Enc(k,m_b,s)$\\
                \> Overwrite the current state $s$ with $s'$ \\
                \> Return $c$
            \end{tabbing}\end{minipage}
        \end{tabular}
    \end{center}
\end{definition}



\end{document}

