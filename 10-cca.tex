\documentclass[11pt]{article}
\usepackage{fullpage}
\usepackage{amsthm}

\usepackage{amsthm,amsmath,amsfonts,amssymb,amstext,enumitem}
\usepackage{latexsym,ifthen,url,rotating,graphicx}
\usepackage{listings}
\usepackage{tikz}
\usetikzlibrary{arrows,shapes,positioning,fit}
\usepackage{graphicx}
\usepackage[font=small,labelfont=bf]{caption}



% --- -----------------------------------------------------------------
% --- Document-specific definitions.
% --- -----------------------------------------------------------------
\lstset{
    columns=fixed,
    literate={—}{{---}}1 {…}{{...}}1
}

\newcommand{\todo}[1]{{\color{red}[TODO:{#1}]}}

\newtheorem{problem}{Problem}
\newtheorem{corollary}{Corollary}
\newtheorem{fact}{Fact}
\newtheorem{exercise}{Exercise}
\newtheorem{theorem}{Theorem}
\newtheorem{definition}{Definition}
\newtheorem{notation}{Notation}
\newtheorem{lemma}{Lemma}
\newtheorem{example}{Example}

\newcommand{\getsr}
  {{\:\stackrel{\raisebox{-2pt}{${\scriptscriptstyle \hspace{0.2em}\$}$}}
   {\leftarrow}\:}}
\newcommand{\points}[1]{\textbf{({#1} pts)}}

\newcommand{\fn}{\footnotesize}
\newcommand{\Colon}{\ : \ }
\newcommand{\st}{\mathsf{state}}
\newcommand{\msgs}{\mathcal{M}}
\newcommand{\ctxts}{\mathcal{C}}
\newcommand{\keys}{\mathcal{K}}
\newcommand{\rands}{\mathcal{R}}
\newcommand{\states}{\mathcal{S}}
\newcommand{\kg}{\mathcal{K}}
\newcommand{\Enc}{\mathsf{Enc}}
\newcommand{\Dec}{\mathsf{Dec}}
\newcommand{\MAC}{\mathrm{MAC}}
\newcommand{\RMAC}{\mathrm{RMAC}}

\newcommand{\pk}{pk}
\newcommand{\sk}{sk}

\newcommand{\calD}{\mathcal{D}}
\newcommand{\calA}{\mathcal{A}}
\newcommand{\calB}{\mathcal{B}}
\newcommand{\AES}{\mathsf{AES}}

\newcommand{\algorithm}[1]{\textbf{Alg} {#1}}

\newcommand{\calO}{\mathcal{O}}

\newcommand{\dlog}{\mathrm{dlog}}

\newcommand{\Adv}{\mathbf{Adv}}
\newcommand{\AdvPRF}[2]{\Adv^{\mathrm{prf}}_{#1}({#2})}
\newcommand{\AdvPRG}[2]{\Adv^{\mathrm{prg}}_{#1}({#2})}
\newcommand{\AdvCPA}[2]{\Adv^{\mathrm{cpa}}_{#1}({#2})}
\newcommand{\AdvCCA}[2]{\Adv^{\mathrm{cca}}_{#1}({#2})}
\newcommand{\AdvKR}[2]{\Adv^{\mathrm{kr}}_{#1}({#2})}
\newcommand{\AdvCKR}[2]{\Adv^{\mathrm{ckr}}_{#1}({#2})}
\newcommand{\AdvRMR}[2]{\Adv^{\mathrm{rmr}}_{#1}({#2})}
\newcommand{\AdvCR}[2]{\Adv^{\mathrm{cr}}_{#1}({#2})}
\newcommand{\AdvUFCMA}[2]{\Adv^{\textrm{uf{-}cma}}_{#1}({#2})}
\newcommand{\AdvDL}[2]{\Adv^{\mathrm{dl}}_{#1}({#2})}

\newcommand{\Exp}{\mathbf{Exp}}
\newcommand{\ExpOW}[1]{\Exp^{\mathrm{ow}}({#1})}
\newcommand{\ExpCKR}[2]{\Exp^{\mathrm{ckr}}_{#1}({#2})}
\newcommand{\ExpRMR}[2]{\Exp^{\mathrm{rmr}}_{#1}({#2})}

\newcommand{\concat}{{\,\|\,}}
\newcommand{\xor}{\oplus}
\newcommand{\bits}{\{0,1\}}

\newcommand{\tcolh}{T^{\mathrm{col}}_h}
\newcommand{\tcolH}{T^{\mathrm{col}}_{H^2}}
\newcommand{\Hcomb}{H^{1\|2}}
\newcommand{\Hxor}{H^{1\oplus2}}

\newcommand{\EXP}{\textrm{EXP}}
\newcommand{\MODEXP}{\textrm{MOD{-}EXP}}
\newcommand{\ADD}{\textrm{ADD}}
\newcommand{\MULTIMODEXP}{\textrm{MULTI{-}MOD{-}EXP}}
\newcommand{\MUL}{\textrm{MUL}}
\newcommand{\MOD}{\textrm{MOD}}

\newcommand{\GG}{\mathbb{G}}
\newcommand{\ZZ}{\mathbb{Z}}

\newcommand{\bK}{\mathbf{K}}
\newcommand{\bof}{\mathbf{f}}
\newcommand{\bU}{\mathbf{U}}
\newcommand{\bM}{\mathbf{M}}
\newcommand{\bC}{\mathbf{C}}

\newcommand{\rvrange}{\mathcal{R}}
\newcommand{\rspace}{\mathcal{C}}

\newcommand{\hatalpha}{\hat{\alpha}}
\newcommand{\hatb}{\hat{b}}
\newcommand{\hatc}{\hat{c}}
\newcommand{\hatm}{\hat{m}}

\newcommand{\barm}{\overline{m}}

\newcommand{\otp}{\mathrm{OTP}}
\newcommand{\des}{\mathrm{DES}}
\newcommand{\twodes}{\mathrm{2DES}}
\newcommand{\threedes}{\mathrm{3DES}}
\newcommand{\threedestwo}{\mathrm{3DES2}}
\newcommand{\aes}{\mathrm{AES}}
\newcommand{\pad}{\mathsf{pad}}
\newcommand{\unpad}{\mathsf{unpad}}
\newcommand{\Func}{\mathrm{Func}}


\newcommand{\Img}{\mathrm{Im}}

\newcommand{\Expt}{\mathbf{Expt}}
\newcommand{\ExptCPA}{\mathbf{Expt}^{\mathrm{cpa}}}
\newcommand{\ExptCCA}{\mathbf{Expt}^{\mathrm{cca}}}
\newcommand{\ExptOTCPA}{\mathbf{Expt}^{\mathrm{1\mbox{-}cpa}}}
\newcommand{\ExptOTCPAone}{\mathbf{Expt}^{\mathrm{1\mbox{-}cpa\mbox{-}1}}}
\newcommand{\ExptOTCPAzero}{\mathbf{Expt}^{\mathrm{1\mbox{-}cpa\mbox{-}0}}}
\newcommand{\AdvOTCPA}[2]{\Adv^{\mathrm{1\mbox{-}cpa}}_{#1}({#2})}
\newcommand{\ExptCPAone}{\mathbf{Expt}^{\mathrm{cpa\mbox{-}1}}}
\newcommand{\ExptCPAzero}{\mathbf{Expt}^{\mathrm{cpa\mbox{-}0}}}

\newcommand{\LR}{\mathrm{LR}}

\newcommand{\Piotp}{\Pi_\mathrm{otp}}
\newcommand{\Encotp}{\Enc_\mathrm{otp}}
\newcommand{\Decotp}{\Dec_\mathrm{otp}}
\newcommand{\bhat}{\hat{b}}
\newcommand{\dict}{\mathtt{dict}}
\newcommand{\Col}{\mathsf{Col}}
% --- -----------------------------------------------------------------
% --- Lecture notes formatting macros
% --- -----------------------------------------------------------------

%
% The following commands set up the lecnum (lecture number)
% counter and make various numbering schemes work relative
% to the lecture number.
%
\newcounter{lecnum}
%\renewcommand{\thepage}{\thelecnum-\arabic{page}}
\renewcommand{\thesection}{\thelecnum.\arabic{section}}
\renewcommand{\theexercise}{\thelecnum.\arabic{exercise}}
\renewcommand{\theexample}{\thelecnum.\arabic{example}}
\renewcommand{\thedefinition}{\thelecnum.\arabic{definition}}
\renewcommand{\theequation}{\thelecnum.\arabic{equation}}
\renewcommand{\thefigure}{\thelecnum.\arabic{figure}}
\renewcommand{\thefact}{\thelecnum.\arabic{fact}}
\renewcommand{\thetable}{\thelecnum.\arabic{table}}


%
% The following macro is used to generate the header.
%
\newcommand{\lecture}[2]{
   %\pagestyle{myheadings}
   %\thispagestyle{plain}
   \newpage
   \setcounter{lecnum}{#1}
   \setcounter{page}{1}
   \noindent
   \begin{center}
   \framebox{
      \vbox{\vspace{2mm}
    \hbox to 6.28in { {\bf CMSC 28400 Introduction to Cryptography
                        \hfill Autumn 2020} }
       \vspace{4mm}
       \hbox to 6.28in { {\Large \hfill #2 \hfill} }
       \vspace{2mm}
       \hbox to 6.28in { {\it Instructor: David Cash} \hfill }
      \vspace{2mm}}
   }
   \end{center}
   %\markboth{Lecture #1: #2}{Lecture #1: #2}
   \vspace*{4mm}
}





% --- -----------------------------------------------------------------
% --- The document starts here.
% --- -----------------------------------------------------------------
\begin{document}
%\lecture{**LECTURE-NUMBER**}{**DATE**}{**LECTURER**}{**SCRIBE**}
\lecture{10}{Notes \#10: Chosen-Ciphertext Security for Encryption}

%\tableofcontents

These notes present and briefly study a definition for \emph{chosen-ciphertext
security} of an encryption scheme. To begin understanding the motivation for
this definition, consider a setting where a server $S_1$ holding a key $k$ is
connected to the internet, and expects to talk to another computer $S_2$ that
also has $k$. But by its very design the internet infrastructure is
\emph{unauthenticated}: Bits are delivered to hosts by routers, and those bits
could claim to be from anywhere. So when our server $S_1$ receives a packet
containing a ciphertext $c$ with the IP address indicating it came from $S_2$,
it can't depend on the IP address alone to ensure the packet is actually
from $S_2$.

Now suppose an adversary controls a router near $S_1$, and injects a packet
that looks like it came from $S_2$.  What can $S_1$ do? In reality it performs
some sort of \emph{cryptographic authentication}, which we discuss in the next
notes, but for now let us approach the question naively. Since $S_1$ has no
idea if the packet is ``real'' or not, \emph{it must run the decryption
algorithm on the received ciphertext $c$}. Perhaps the decryption algorithm
returns garbage, or perhaps $c$ has been crafted to return something more
controlled. In any case, $S_1$ \emph{must react to $c$ in some way that is
observable to the adversary}. Perhaps $S_1$ closes the connection, or maybe it
attempts to send an error code, or maybe it even sends the next message in a
protocol.

In summary, the ability to inject one or more ciphertexts $c$ gives an
adversary the power to carry out attacks fitting the following template:
\begin{enumerate}
    \item Cook up some ciphertext $\hatc$.
    \item Have a receiver run $\hatm \gets \Dec(k,\hatc)$.
    \item Observe how the receiver reacts to the message $\hatm$.
\end{enumerate}
These are called \emph{chosen-ciphertext attacks}, or CCAs.
It's hard to intuit if CCAs like this could ever be damaging. Let's look
at an example that highlights how things might go catastrophically wrong.
\begin{example}
    Let $\Pi = (\Enc,\Dec)$ be the (deterministic) OTP. Suppose a server
    uses $\Pi$ to decrypt messages it receives. 
    Suppose further that if the server decrypts a ciphertext
    $c$ and obtains a message $m$
    that does not have the correct format (e.g. is not proper
    HTTP), then it replies with $m$ in the clear to the sender and requests
    a retransmission.

    Here is a CCA that recovers the secret key from the server:
    \begin{enumerate}
        \item Set $\hatc$ to a random string, and send it to the server.
        \item The server computes $\hatm\gets\Dec(k,\hatc)$. It's likely
            that $\hatm=k\oplus \hatc$ will
            not be a properly-formatted message, so the server replies with
            $\hatm$.
        \item The attacker computes $k = \hatc\oplus\hatm$.
    \end{enumerate}
\end{example}
The above example is extreme, but bugs like this have been known to leak
keys; One problem in Project 2 gives a more practical version.


So we want a definition that ensures CCAs won't be damaging. We certainly want
that keys won't be leaked, but we'd like to ensure even more, akin to the
guarantees we got from CPA security. Thus we will aim for a left/right
indistinguishability definition like CPA, but we need to give the adversary
some power similar to the template above. The types of reactions a server might
have to a ciphertext $\hatc$ are quite wide, and we can't hope to enumerate
them. Instead, we'll go completely overkill, and \emph{give the adversary the
ability to decrypt ciphertexts as it likes}. We do this formally with another
oracle. That way, since the adversary can learn how any $\hatc$ to decrypts to
some $\hatm$, it can also learn about any possible reaction a server might
have. And if a scheme resists these attacks, then it should resist more
realistic attacks fitting the template above.

There is one wrinkle, however: We must stop the adversary from using its
decryption ability to trivially win the experiment. After all, if the adversary
is trying to learn what message was encrypted ($m_0$ or $m_1$), decryption of
arbitrary ciphertexts will allow it easily figure that out. Thus we need an
extra rule saying ``free win'' ciphertexts are not allowed to be decrypted.

The formal definition follows. It fits the template we've had so far, but with
an extra oracle.
\begin{definition}
    Let $\Pi = (\Enc,\Dec)$ be a randomized encryption scheme with key-space
    $\keys$, message-space $\msgs$, randomness-space $\rands$, and
    ciphertext-space $\ctxts$.  Assume that $\msgs\subseteq \bits^*$.  Let
    $\calA$ be an algorithm. Define algorithm $\ExptCCA_\Pi(\calA)$ as
    \begin{center}
    \begin{tabular}{c}
        \begin{minipage}{2in}\begin{tabbing}
            123\=123\=\kill
            \underline{\algorithm{$\ExptCCA_\Pi(\calA)$}} \\[2pt]
            \fn01 \> Pick $k\getsr \keys, b\getsr \bits$\\
            \fn02 \> Run $\calA^{\LR_{k,b}(\cdot,\cdot),\Dec(k,\cdot)}$, where
            the first oracle is given below. Eventually $\calA$ halts with
            output $\hatb$\\
            \fn03 \> If $\calA$ ever queried $\Dec(k,\cdot)$ at a ciphertext
            $c$ previously output by $\LR_{k,b}(\cdot,\cdot)$: Output $\bot$. \\
            \fn04\> If $\hatb = b$: Output 1\\
            \fn05 \> Else: Output 0\\
            \\
            \underline{Oracle $\LR_{k,b}(m_0,m_1)$} \\
            \> If $m_0,m_1$ are not the same length: Return $\bot$\\
            \> Pick $r \getsr \rands$\\
            \> Compute $c \gets \Enc(k,m_b,r)$\\
            \> Return $c$
        \end{tabbing}\end{minipage}
    \end{tabular}
    \end{center}
    Define the \emph{CCA advantage of $\calA$ against $\Pi$} as
    \[
        \AdvCCA{\Pi}{\calA} =
        \left|\Pr[\ExptCCA_\Pi(\calA) = 1] - \frac{1}{2}\right|
    \]
\end{definition}
This definition is exactly the CPA definition, except for the $\Dec(k,\cdot)$
oracle and for line $03$, which enforces the ``free win'' rule (see the next
paragraph). Recall that with an oracle, an algorithm can submit whatever it
likes, and submit as many queries as it likes (bounded only by its runtime).
Thus an adversary above can ask for decryptions of many ciphertexts (millions),
each of which it gets to form. They could be malformed, all-zeros, or whatever
else might help a break. The idea is that all of the decryptions should not
help the adversary break any of the ``challenge'' ciphertexts from the $\LR$
oracle.


Note that the game is easy to win if we omit the rule on line $03$;
We leave this as an exercise to check.
\begin{exercise}
    Suppose the definition above omitted line $03$.  Assume $\Pi$ has more than
    one allowed message, i.e.  $|\msgs| > 1$. Give an efficient adversary
    $\calA$ such that
    \[
        \AdvCCA{\Pi}{\calA} = 1/2.
    \]
\end{exercise}

Building encryption from a block cipher that resists chosen-ciphertext attacks
is considerably harder than chosen-plaintext attacks. In fact, we won't even
try in these notes and will defer it to the next set. The following example is
meant to highlight the challenge.
\begin{example}
Let $E:\bits^n\times\bits^\ell\to\bits^\ell$ be a block cipher. 
Define a randomized encryption encryption scheme 
    $\Pi=(\Enc,\Dec)$ with key-space $\bits^n$, message-
    and randomness-spaces $\msgs=\rands=\bits^\ell$, and ciphertext-space
    $\ctxts=\bits^\ell\times\bits^\ell$ by
    \[
        \Enc(k,m,r)=(r,E(k,r)\oplus m)
    \]
    and $\Dec(k,(r,c)) = E(k,r)\oplus c$.

    Previously we showed that breaking the CPA-security of $\Pi$ required
    breaking the PRF-security of $E$. We now give an efficient adversary
    $\calA$ that strongly breaks the CCA-security of $\Pi$, satisfying
    \[
        \AdvCCA{\Pi}{\calA} = 1/2.
    \]
    This shows the construction is not at all secure against chosen-ciphertext
    attack.
    The adversary works follows:

    \begin{center}
        \begin{tabular}{c}
            \begin{minipage}{2in}\begin{tabbing}
                123\=123\=\kill
                \underline{Adversary $\calA^{\LR_{k,b}(m_0,m_1),\Dec(k,\cdot)}$} \\[2pt]
                \> Let $m_0\neq m_1\in\msgs$, with $m_0\neq 0^\ell$. 
                Query $(r,c) \gets \LR_{k,b}(m_0,m_1)$.\\
                \> Let $c' = c \oplus m_0$. Query $m' \gets \Dec(k,(r,c'))$\\
                \> If $m' = 0^\ell$: Output $0$\\
                \> Else: Output $1$.
            \end{tabbing}\end{minipage}
        \end{tabular}
    \end{center}

    Let us check that the adversary always win $\ExptCCA_\Pi(\calA)$.
    First, $\calA$ will never auto-lose the game, because $(r,c)$ is never
    queried to the decryption oracle (only $(r,c')$ is, but $(r,c')\neq (r,c)$
    because $c'\neq c$, which is where we use that $m_0\neq 0^\ell$). Now
    suppose $b=0$, so $(r,c) = (r,E(k,r)\oplus m_0)$. Then the ciphertext
    $(r,c')$ will decrypt as
    \[
        c' \oplus E(k,r) 
        = c\oplus m_0 \oplus E(k,r) 
        = E(k,r)\oplus m_0 \oplus m_0 \oplus E(k,r) 
        = 0^\ell,
    \]
    and $\calA$ will output $0$. Otherwise, if $b=1$, $(r,c')$ will decrypt
    as
    \[
        c' \oplus E(k,r) 
        = c\oplus m_0 \oplus E(k,r) 
        = E(k,r)\oplus m_1 \oplus m_0 \oplus E(k,r) 
        = m_1\oplus m_0 \neq 0^\ell,
    \]
    because $m_0\neq m_1$. Thus $\calA$ will output $1$. This shows that
    $\calA$ always wins and thus has the desired advantage.
\end{example}
Intuitively, the problem is \emph{malleability of ciphertexts}: The adversary
$\calA$, without knowing the key, is able to modify a ciphertext $c$ into a new
ciphertext $c'$ containing a message related to the original one. This new
ciphertext can be submitted for decryption (because it is different), and
the response can be used to win the experiment.

Don't overthink the next exercise: It's trivial.
\begin{exercise}
    Let $\Pi$ be an encryption scheme. Show that for all $\calA$ there exists
    a $\calB$ running in the same time such that
    \[
        \AdvCPA{\Pi}{\calA} \leq \AdvCCA{\Pi}{\calB}.
    \]
    How are CCA and CPA security related?
\end{exercise}

Unfortunately, CTR and CBC modes are not CCA-secure. We'll need new tools
instead.
\begin{exercise}
    Let $\Pi$ be either AES-CTR or AES-CBC, defined in the previous notes.
    For each one, find an efficient
    $\calA$ such that
    \[
        \AdvCCA{\Pi}{\calA} = 1/2.
    \]
\end{exercise}

\end{document}

