\documentclass[11pt]{article}
\usepackage{fullpage}
\usepackage{amsthm}

\usepackage{amsthm,amsmath,amsfonts,amssymb,amstext,enumitem}
\usepackage{latexsym,ifthen,url,rotating,graphicx}
\usepackage{listings}
\usepackage{tikz}
\usetikzlibrary{arrows,shapes,positioning,fit}
\usepackage{graphicx}
\usepackage[font=small,labelfont=bf]{caption}



% --- -----------------------------------------------------------------
% --- Document-specific definitions.
% --- -----------------------------------------------------------------
\lstset{
    columns=fixed,
    literate={—}{{---}}1 {…}{{...}}1
}

\newcommand{\todo}[1]{{\color{red}[TODO:{#1}]}}

\newtheorem{problem}{Problem}
\newtheorem{corollary}{Corollary}
\newtheorem{fact}{Fact}
\newtheorem{exercise}{Exercise}
\newtheorem{theorem}{Theorem}
\newtheorem{definition}{Definition}
\newtheorem{notation}{Notation}
\newtheorem{lemma}{Lemma}
\newtheorem{example}{Example}

\newcommand{\getsr}
  {{\:\stackrel{\raisebox{-2pt}{${\scriptscriptstyle \hspace{0.2em}\$}$}}
   {\leftarrow}\:}}
\newcommand{\points}[1]{\textbf{({#1} pts)}}

\newcommand{\fn}{\footnotesize}
\newcommand{\Colon}{\ : \ }
\newcommand{\st}{\mathsf{state}}
\newcommand{\msgs}{\mathcal{M}}
\newcommand{\ctxts}{\mathcal{C}}
\newcommand{\keys}{\mathcal{K}}
\newcommand{\rands}{\mathcal{R}}
\newcommand{\states}{\mathcal{S}}
\newcommand{\kg}{\mathcal{K}}
\newcommand{\Enc}{\mathsf{Enc}}
\newcommand{\Dec}{\mathsf{Dec}}
\newcommand{\MAC}{\mathrm{Mac}}
\newcommand{\Vrfy}{\mathrm{Vrfy}}
\newcommand{\RMAC}{\mathrm{RMAC}}
\newcommand{\tags}{\mathcal{T}}
\newcommand{\testeq}{\stackrel{?}{=}}
\newcommand{\win}{\mathsf{win}}
\newcommand{\leng}{\mathsf{len}}

\newcommand{\pk}{pk}
\newcommand{\sk}{sk}

\newcommand{\calD}{\mathcal{D}}
\newcommand{\calA}{\mathcal{A}}
\newcommand{\calE}{\mathcal{E}}
\newcommand{\calB}{\mathcal{B}}
\newcommand{\AES}{\mathsf{AES}}

\newcommand{\algorithm}[1]{\textbf{Alg} {#1}}

\newcommand{\calO}{\mathcal{O}}

\newcommand{\dlog}{\mathrm{dlog}}

\newcommand{\Adv}{\mathbf{Adv}}
\newcommand{\AdvPRF}[2]{\Adv^{\mathrm{prf}}_{#1}({#2})}
\newcommand{\AdvPRG}[2]{\Adv^{\mathrm{prg}}_{#1}({#2})}
\newcommand{\AdvCPA}[2]{\Adv^{\mathrm{cpa}}_{#1}({#2})}
\newcommand{\AdvCCA}[2]{\Adv^{\mathrm{cca}}_{#1}({#2})}
\newcommand{\AdvKR}[2]{\Adv^{\mathrm{kr}}_{#1}({#2})}
\newcommand{\AdvCKR}[2]{\Adv^{\mathrm{ckr}}_{#1}({#2})}
\newcommand{\AdvRMR}[2]{\Adv^{\mathrm{rmr}}_{#1}({#2})}
\newcommand{\AdvCR}[2]{\Adv^{\mathrm{cr}}_{#1}({#2})}
\newcommand{\AdvUFCMA}[2]{\Adv^{\textrm{uf{-}cma}}_{#1}({#2})}
\newcommand{\AdvDL}[2]{\Adv^{\mathrm{dl}}_{#1}({#2})}

\newcommand{\Exp}{\mathbf{Exp}}
\newcommand{\ExpOW}[1]{\Exp^{\mathrm{ow}}({#1})}
\newcommand{\ExpCKR}[2]{\Exp^{\mathrm{ckr}}_{#1}({#2})}
\newcommand{\ExpRMR}[2]{\Exp^{\mathrm{rmr}}_{#1}({#2})}

\newcommand{\concat}{{\,\|\,}}
\newcommand{\xor}{\oplus}
\newcommand{\bits}{\{0,1\}}

\newcommand{\tcolh}{T^{\mathrm{col}}_h}
\newcommand{\tcolH}{T^{\mathrm{col}}_{H^2}}
\newcommand{\Hcomb}{H^{1\|2}}
\newcommand{\Hxor}{H^{1\oplus2}}

\newcommand{\EXP}{\textrm{EXP}}
\newcommand{\MODEXP}{\textrm{MOD{-}EXP}}
\newcommand{\ADD}{\textrm{ADD}}
\newcommand{\MULTIMODEXP}{\textrm{MULTI{-}MOD{-}EXP}}
\newcommand{\MUL}{\textrm{MUL}}
\newcommand{\MOD}{\textrm{MOD}}

\newcommand{\GG}{\mathbb{G}}
\newcommand{\ZZ}{\mathbb{Z}}

\newcommand{\bK}{\mathbf{K}}
\newcommand{\bof}{\mathbf{f}}
\newcommand{\bU}{\mathbf{U}}
\newcommand{\bM}{\mathbf{M}}
\newcommand{\bC}{\mathbf{C}}

\newcommand{\rvrange}{\mathcal{R}}
\newcommand{\rspace}{\mathcal{C}}

\newcommand{\hatalpha}{\hat{\alpha}}
\newcommand{\hatb}{\hat{b}}
\newcommand{\hatc}{\hat{c}}
\newcommand{\hatt}{\hat{t}}
\newcommand{\hatm}{\hat{m}}

\newcommand{\barm}{\overline{m}}

\newcommand{\otp}{\mathrm{OTP}}
\newcommand{\des}{\mathrm{DES}}
\newcommand{\twodes}{\mathrm{2DES}}
\newcommand{\threedes}{\mathrm{3DES}}
\newcommand{\threedestwo}{\mathrm{3DES2}}
\newcommand{\aes}{\mathrm{AES}}
\newcommand{\pad}{\mathsf{pad}}
\newcommand{\unpad}{\mathsf{unpad}}
\newcommand{\Func}{\mathrm{Func}}
\newcommand{\cbcmacpad}{\mathsf{pad}_{\mathrm{cbc}}}


\newcommand{\Img}{\mathrm{Im}}

\newcommand{\Expt}{\mathbf{Expt}}
\newcommand{\ExptCPA}{\mathbf{Expt}^{\mathrm{cpa}}}
\newcommand{\ExptCCA}{\mathbf{Expt}^{\mathrm{cca}}}
\newcommand{\ExptOTCPA}{\mathbf{Expt}^{\mathrm{1\mbox{-}cpa}}}
\newcommand{\ExptOTCPAone}{\mathbf{Expt}^{\mathrm{1\mbox{-}cpa\mbox{-}1}}}
\newcommand{\ExptOTCPAzero}{\mathbf{Expt}^{\mathrm{1\mbox{-}cpa\mbox{-}0}}}
\newcommand{\AdvOTCPA}[2]{\Adv^{\mathrm{1\mbox{-}cpa}}_{#1}({#2})}
\newcommand{\ExptCPAone}{\mathbf{Expt}^{\mathrm{cpa\mbox{-}1}}}
\newcommand{\ExptCPAzero}{\mathbf{Expt}^{\mathrm{cpa\mbox{-}0}}}
\newcommand{\ExptUF}{\mathbf{Expt}^{\mathrm{uf}}}
\newcommand{\AdvUF}[2]{\Adv^{\mathrm{uf}}_{#1}({#2})}

\newcommand{\LR}{\mathrm{LR}}

\newcommand{\Piotp}{\Pi_\mathrm{otp}}
\newcommand{\Encotp}{\Enc_\mathrm{otp}}
\newcommand{\Decotp}{\Dec_\mathrm{otp}}
\newcommand{\bhat}{\hat{b}}
\newcommand{\dict}{\mathtt{dict}}
\newcommand{\Col}{\mathsf{Col}}
% --- -----------------------------------------------------------------
% --- Lecture notes formatting macros
% --- -----------------------------------------------------------------

%
% The following commands set up the lecnum (lecture number)
% counter and make various numbering schemes work relative
% to the lecture number.
%
\newcounter{lecnum}
%\renewcommand{\thepage}{\thelecnum-\arabic{page}}
\renewcommand{\thesection}{\thelecnum.\arabic{section}}
\renewcommand{\theexercise}{\thelecnum.\arabic{exercise}}
\renewcommand{\theexample}{\thelecnum.\arabic{example}}
\renewcommand{\thedefinition}{\thelecnum.\arabic{definition}}
\renewcommand{\theequation}{\thelecnum.\arabic{equation}}
\renewcommand{\thefigure}{\thelecnum.\arabic{figure}}
\renewcommand{\thefact}{\thelecnum.\arabic{fact}}
\renewcommand{\thetable}{\thelecnum.\arabic{table}}


%
% The following macro is used to generate the header.
%
\newcommand{\lecture}[2]{
   %\pagestyle{myheadings}
   %\thispagestyle{plain}
   \newpage
   \setcounter{lecnum}{#1}
   \setcounter{page}{1}
   \noindent
   \begin{center}
   \framebox{
      \vbox{\vspace{2mm}
    \hbox to 6.28in { {\bf CMSC 28400 Introduction to Cryptography
                        \hfill Autumn 2020} }
       \vspace{4mm}
       \hbox to 6.28in { {\Large \hfill #2 \hfill} }
       \vspace{2mm}
       \hbox to 6.28in { {\it Instructor: David Cash} \hfill }
      \vspace{2mm}}
   }
   \end{center}
   %\markboth{Lecture #1: #2}{Lecture #1: #2}
   \vspace*{4mm}
}





% --- -----------------------------------------------------------------
% --- The document starts here.
% --- -----------------------------------------------------------------
\begin{document}
%\lecture{**LECTURE-NUMBER**}{**DATE**}{**LECTURER**}{**SCRIBE**}
\lecture{15}{Notes \#15: Analysis of Floyd's Cycle-Finding Algorithm}

For all the claims below, let $f:X\to X$ be a function from some finite set $X$
to itself and let $a_0 \in X$. Define a sequence of points in $X$ by $a_1 =
f(a_0)$, $a_2 = f(a_1)=f(f(a_0))$, and in general for any positive integer $k$,
\[
    a_k = f^{(k)}(a_0),
\]
where $f^{(k)} = f(f(\cdots f(a_0)\cdots))$ is the function $f$ iterated $k$
times.

\begin{lemma}
   There exist $I,J$ non-negative integers such that $I<J$ and $a_I = a_J$.
\end{lemma}
\begin{proof}
    This follows by the pigeonhole principle, since $X$ is a finite set
    and the sequence is infinite.
\end{proof}

\begin{lemma}\label{lem:period}
    Let $J$ be the smallest non-negative integer such that 
    \[
        a_J \in \{a_0,\ldots,a_{J-1}\}.
    \]
    This choice of $J$ implies there is a unique integer $0 \leq I < J$ such
    that $a_I = a_J$.  Let $\Delta = J-I$.  Then for all $k\geq I$,
    \[
        a_k = a_{I+[k-I \bmod \Delta]}.
    \]
    Moreover, for $i,j\geq I$,
    we have that $a_i = a_j$ if and only if $i \equiv j \bmod \Delta$.
\end{lemma}
\begin{proof}
    Intuitively, the picture from the video suggests
    that the sequence is going around a circle of size $\Delta$ (and this
    is definitely the way to understand the lemma!).  

    First we show that for any $q\geq 0$, 
    \[
        a_{I+q\Delta} = a_I.
    \]
    Use induction on $q$. For $q=0$ this is trivial. For $q>0$,
    \begin{align*}
        a_{I+q\Delta}  
        & = f^{I+q\Delta}(a_0) \\
        & = f^{(q-1)\Delta}(f^{I+\Delta}(a_0))\\
        & = f^{(q-1)\Delta}(a_I)\\
        & = a_I.
    \end{align*}
    The first equality is by definition, and the second is just rewriting
    the applications of $f$ into two batches. The third equality uses
    the assumption that $a_J = a_{I+\Delta} = a_I$. The last equality
    is by the inductive hypothesis.

    Let $k\geq I$. By division with remainder, we can write
    $k-I = q\Delta + r$ with $0\leq r < \Delta$.
    Then since $a_I = a_J$,
    \[
        a_k 
        = a_{I + q\Delta + r}
        = f^{I + q\Delta + r}(a_0)
        = f^{r}(f^{I+q\Delta}(a_0))
        = f^{r}(a_I)
        = a_{I+r}.
    \]
    This proves the first part of the Lemma, since $r = [k-I\bmod \Delta]$.

    Now for the second part. The first part already shows that if
    $i\equiv j \bmod \Delta$ then 
    \[
        a_i 
        = a_{I+[i-I\bmod \Delta]}
        =a_{I+[j-I\bmod \Delta]}
        = a_j.
    \]
    For the reverse direction, suppose $a_i = a_j$. Then $a_{I+[i-I\bmod
    \Delta]} = a_{I+[j-I\bmod \Delta]}$ by the first part. But the elements
    \[
        a_I,a_{I+1},\ldots,a_{I+\Delta-1}
    \]
    are all distinct, so we must have $[i-I\bmod\Delta] = [j-I\bmod\Delta]$,
    which implies $i\equiv j \bmod \Delta$.
\end{proof}

\begin{lemma}
    Let $I,J,\Delta$ be as in Lemma~\ref{lem:period}. 
    Then there exists an integer $i$ less than $J$ such that $a_i=a_{2i}$.
    Moreover, any $i$ satisfying $a_i=a_{2i}$ must be a multiple of
    $\Delta$.
\end{lemma}
\begin{proof}
    %Take $\Delta = J-I$; 
    %Then $0 < \Delta < J$. 
    We show existence first.
    We claim there exists an
    integer $i$, $I\leq i < J$, that is a multiple of $\Delta$. To see
    this, use division with remainder to conclude there exist integers
    $q,r$ such that
    \[
        I = q\Delta + r, \quad 0\leq r < \Delta.
    \]
    Then if $r=0$ we can take $i = q\Delta$, and otherwise if $r>0$ we can take
    $i=(q+1)\Delta$. In either case $i$ is clearly a multiple of $\Delta$ and
    satisfies $i\geq I$.  We
    claim also that in either case $i<J$. If $i=q\Delta = I$ then this
    is true because we assumed $I<J$. If $i=(q+1)\Delta$ then
    \[
        i = (q+1)\Delta = q\Delta + \Delta = q\Delta + r + (\Delta-r) 
        = I + (\Delta - r) < I + \Delta = J.
    \]

    Now assume we have an $I\leq i<J$ that is a multiple of $\Delta$.
    The previous lemma 
    (applied repeatedly) implies that
    \[
        a_{2i} = a_{i+i} = a_i,
    \]
    since adding a multiple of $\Delta$ leaves the value unchanged (and
    $i$ is such a multiple!).

    Now suppose $a_i=a_{2i}$. Then $i\geq I$ and $2i\geq J$ since we
    chose $I,J$ to be the indexes of the first repetition.
\end{proof}

\begin{lemma}
    Let $I,J$ be the smallest non-negative integers such that $I<J$ and $a_I =
    a_J$ (that is, $J$ is the index of the first repeated value, and $I<J$ is
    the index of the previous appearance).  If $i,\delta$ are non-negative
    integers that satisfy $a_i = a_{i+\delta}$, then $\delta$ must be a
    multiple of $\Delta = J-I$.
\end{lemma}
\begin{proof}
    Suppose $\delta$ is not a multiple of $\Delta$.
    By Lemma~\ref{lem:plusdelta}, we can assume
\end{proof}

\begin{lemma}
    Let $I,J$ be the smallest non-negative integers such that $I<J$ and $a_I =
    a_J$.  
    If $i$ is a non-negative integer that satisfies $a_i = a_{2i}$, then we
    also have
    \[
        a_{i+I} = a_I.
    \]
\end{lemma}

\end{document}

