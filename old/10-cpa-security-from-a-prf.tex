\documentclass[11pt]{article}
\usepackage{fullpage}
\usepackage{amsthm}

\usepackage{amsthm,amsmath,amsfonts,amssymb,amstext,enumitem}
\usepackage{latexsym,ifthen,url,rotating,graphicx}
\usepackage{listings}
\usepackage{tikz}
\usetikzlibrary{arrows,shapes,positioning,fit}
\usepackage{graphicx}
\usepackage[font=small,labelfont=bf]{caption}



% --- -----------------------------------------------------------------
% --- Document-specific definitions.
% --- -----------------------------------------------------------------
\lstset{
    columns=fixed,
    literate={—}{{---}}1 {…}{{...}}1
}

\newcommand{\todo}[1]{{\color{red}[TODO:{#1}]}}

\newtheorem{problem}{Problem}
\newtheorem{corollary}{Corollary}
\newtheorem{fact}{Fact}
\newtheorem{exercise}{Exercise}
\newtheorem{theorem}{Theorem}
\newtheorem{definition}{Definition}
\newtheorem{notation}{Notation}
\newtheorem{lemma}{Lemma}
\newtheorem{example}{Example}

\newcommand{\getsr}
  {{\:\stackrel{\raisebox{-2pt}{${\scriptscriptstyle \hspace{0.2em}\$}$}}
   {\leftarrow}\:}}
\newcommand{\points}[1]{\textbf{({#1} pts)}}

\newcommand{\fn}{\footnotesize}
\newcommand{\Colon}{\ : \ }
\newcommand{\st}{\mathsf{state}}
\newcommand{\msgs}{\mathcal{M}}
\newcommand{\ctxts}{\mathcal{C}}
\newcommand{\keys}{\mathcal{K}}
\newcommand{\rands}{\mathcal{R}}
\newcommand{\states}{\mathcal{S}}
\newcommand{\kg}{\mathcal{K}}
\newcommand{\Enc}{\mathsf{Enc}}
\newcommand{\Dec}{\mathsf{Dec}}
\newcommand{\MAC}{\mathrm{MAC}}
\newcommand{\RMAC}{\mathrm{RMAC}}

\newcommand{\pk}{pk}
\newcommand{\sk}{sk}

\newcommand{\calD}{\mathcal{D}}
\newcommand{\calA}{\mathcal{A}}
\newcommand{\calB}{\mathcal{B}}
\newcommand{\AES}{\mathsf{AES}}

\newcommand{\algorithm}[1]{\textbf{Alg} {#1}}

\newcommand{\calO}{\mathcal{O}}

\newcommand{\dlog}{\mathrm{dlog}}

\newcommand{\Adv}{\mathbf{Adv}}
\newcommand{\AdvPRF}[2]{\Adv^{\mathrm{prf}}_{#1}({#2})}
\newcommand{\AdvPRG}[2]{\Adv^{\mathrm{prg}}_{#1}({#2})}
\newcommand{\AdvCPA}[2]{\Adv^{\mathrm{cpa}}_{#1}({#2})}
\newcommand{\AdvCCA}[2]{\Adv^{\mathrm{ind{-}cca}}_{#1}({#2})}
\newcommand{\AdvKR}[2]{\Adv^{\mathrm{kr}}_{#1}({#2})}
\newcommand{\AdvCKR}[2]{\Adv^{\mathrm{ckr}}_{#1}({#2})}
\newcommand{\AdvRMR}[2]{\Adv^{\mathrm{rmr}}_{#1}({#2})}
\newcommand{\AdvCR}[2]{\Adv^{\mathrm{cr}}_{#1}({#2})}
\newcommand{\AdvUFCMA}[2]{\Adv^{\textrm{uf{-}cma}}_{#1}({#2})}
\newcommand{\AdvDL}[2]{\Adv^{\mathrm{dl}}_{#1}({#2})}

\newcommand{\Exp}{\mathbf{Exp}}
\newcommand{\ExpOW}[1]{\Exp^{\mathrm{ow}}({#1})}
\newcommand{\ExpCKR}[2]{\Exp^{\mathrm{ckr}}_{#1}({#2})}
\newcommand{\ExpRMR}[2]{\Exp^{\mathrm{rmr}}_{#1}({#2})}

\newcommand{\concat}{{\,\|\,}}
\newcommand{\xor}{\oplus}
\newcommand{\bits}{\{0,1\}}

\newcommand{\tcolh}{T^{\mathrm{col}}_h}
\newcommand{\tcolH}{T^{\mathrm{col}}_{H^2}}
\newcommand{\Hcomb}{H^{1\|2}}
\newcommand{\Hxor}{H^{1\oplus2}}

\newcommand{\EXP}{\textrm{EXP}}
\newcommand{\MODEXP}{\textrm{MOD{-}EXP}}
\newcommand{\ADD}{\textrm{ADD}}
\newcommand{\MULTIMODEXP}{\textrm{MULTI{-}MOD{-}EXP}}
\newcommand{\MUL}{\textrm{MUL}}
\newcommand{\MOD}{\textrm{MOD}}

\newcommand{\GG}{\mathbb{G}}
\newcommand{\ZZ}{\mathbb{Z}}

\newcommand{\bK}{\mathbf{K}}
\newcommand{\bof}{\mathbf{f}}
\newcommand{\bU}{\mathbf{U}}
\newcommand{\bM}{\mathbf{M}}
\newcommand{\bC}{\mathbf{C}}

\newcommand{\rvrange}{\mathcal{R}}
\newcommand{\rspace}{\mathcal{C}}

\newcommand{\hatalpha}{\hat{\alpha}}
\newcommand{\hatb}{\hat{b}}

\newcommand{\barm}{\overline{m}}

\newcommand{\otp}{\mathrm{OTP}}
\newcommand{\des}{\mathrm{DES}}
\newcommand{\twodes}{\mathrm{2DES}}
\newcommand{\threedes}{\mathrm{3DES}}
\newcommand{\threedestwo}{\mathrm{3DES2}}
\newcommand{\aes}{\mathrm{AES}}
\newcommand{\pad}{\mathsf{pad}}
\newcommand{\unpad}{\mathsf{unpad}}
\newcommand{\Func}{\mathrm{Func}}


\newcommand{\Img}{\mathrm{Im}}

\newcommand{\Expt}{\mathbf{Expt}}
\newcommand{\ExptCPA}{\mathbf{Expt}^{\mathrm{cpa}}}
\newcommand{\ExptCCA}{\mathbf{Expt}^{\mathrm{cca}}}
\newcommand{\ExptOTCPA}{\mathbf{Expt}^{\mathrm{1\mbox{-}cpa}}}
\newcommand{\ExptOTCPAone}{\mathbf{Expt}^{\mathrm{1\mbox{-}cpa\mbox{-}1}}}
\newcommand{\ExptOTCPAzero}{\mathbf{Expt}^{\mathrm{1\mbox{-}cpa\mbox{-}0}}}
\newcommand{\AdvOTCPA}[2]{\Adv^{\mathrm{1\mbox{-}cpa}}_{#1}({#2})}
\newcommand{\ExptCPAone}{\mathbf{Expt}^{\mathrm{cpa\mbox{-}1}}}
\newcommand{\ExptCPAzero}{\mathbf{Expt}^{\mathrm{cpa\mbox{-}0}}}

\newcommand{\LR}{\mathrm{LR}}

\newcommand{\Piotp}{\Pi_\mathrm{otp}}
\newcommand{\Encotp}{\Enc_\mathrm{otp}}
\newcommand{\Decotp}{\Dec_\mathrm{otp}}
\newcommand{\bhat}{\hat{b}}
\newcommand{\dict}{\mathtt{dict}}
\newcommand{\Col}{\mathsf{Col}}
% --- -----------------------------------------------------------------
% --- Lecture notes formatting macros
% --- -----------------------------------------------------------------

%
% The following commands set up the lecnum (lecture number)
% counter and make various numbering schemes work relative
% to the lecture number.
%
\newcounter{lecnum}
%\renewcommand{\thepage}{\thelecnum-\arabic{page}}
\renewcommand{\thesection}{\thelecnum.\arabic{section}}
\renewcommand{\theexercise}{\thelecnum.\arabic{exercise}}
\renewcommand{\theexample}{\thelecnum.\arabic{example}}
\renewcommand{\thedefinition}{\thelecnum.\arabic{definition}}
\renewcommand{\theequation}{\thelecnum.\arabic{equation}}
\renewcommand{\thefigure}{\thelecnum.\arabic{figure}}
\renewcommand{\thefact}{\thelecnum.\arabic{fact}}
\renewcommand{\thetable}{\thelecnum.\arabic{table}}


%
% The following macro is used to generate the header.
%
\newcommand{\lecture}[2]{
   %\pagestyle{myheadings}
   %\thispagestyle{plain}
   \newpage
   \setcounter{lecnum}{#1}
   \setcounter{page}{1}
   \noindent
   \begin{center}
   \framebox{
      \vbox{\vspace{2mm}
    \hbox to 6.28in { {\bf CMSC 28400 Introduction to Cryptography
                        \hfill Autumn 2019} }
       \vspace{4mm}
       \hbox to 6.28in { {\Large \hfill #2 \hfill} }
       \vspace{2mm}
       \hbox to 6.28in { {\it Instructor: David Cash} \hfill }
      \vspace{2mm}}
   }
   \end{center}
   %\markboth{Lecture #1: #2}{Lecture #1: #2}
   \vspace*{4mm}
}





% --- -----------------------------------------------------------------
% --- The document starts here.
% --- -----------------------------------------------------------------
\begin{document}
%\lecture{**LECTURE-NUMBER**}{**DATE**}{**LECTURER**}{**SCRIBE**}
\lecture{10}{Notes \#10: CPA Security from a PRF}

%\tableofcontents

\noindent\hrulefill
\bigskip

\begin{theorem}
Let $E:\bits^n\times\bits^\ell\to\bits^\ell$ be a block cipher. 
Define a randomized encryption encryption scheme 
    $\Pi=(\Enc,\Dec)$ with key-space $\bits^n$, message-
    and randomness-spaces $\msgs=\rands=\bits^\ell$, and ciphertext-space
    $\ctxts=\bits^\ell\times\bits^\ell$ by
    \[
        \Enc(k,m,r)=(r,E(k,r)\oplus m)
    \]
    and $\Dec(k,(r,c)) = E(k,r)\oplus c$. Then for every $\calA$ there
    exists a $\calB$, running in about the same time as $\calA$, such
    that
    \[
        \AdvCPA{\Pi}{\calA} \leq \AdvPRF{E}{\calB} + \Col(2^\ell,q),
    \]
    where $q$ is the number queries issued by $\calA$.
\end{theorem}
\begin{proof}
    We construct the needed adversary $\calB$, which has access to an oracle
    $\calO$ that is either $E$ with a random key, or a random function $f$.
    The high-level idea is similar to the reduction to PRG security for
    one-time CPA, in that $\calB$ will need to ``simulate'' the CPA game for
    $\calA$. The main differences for the simulation are that $\calB$ has an
    oracle instead of an input, and the scheme $\Pi$ under consideration is
    randomized rather than deterministic. The similarity is that we want
    $\calB$, when connected to the oracle $E(\bK,\cdot)$, to simulate
    \emph{exactly} $\ExptCPA_{\Pi}(\calA)$, while if the oracle is
    $\bof(\cdot)$ then we want to simulate a game that $\calA$ can't win
    too often.

    Our $\calB$ has access to an oracle $\calO(\cdot)$ mapping $\bits^\ell$ to
    $\bits^\ell$; It is trying to determine if $\calO(\cdot)=E(\bK,\cdot)$
    or $\calO(\cdot)=\bof(\cdot)$.

    The adversary $\calB^\calO$
    picks a bit $b$ and runs $\calA$. 
    When $\calA$ queries $\LR_{k,b}(m_0,m_1)$,
    $\calB$ picks $r\in\bits^\ell$ and returns
    \[
        (r, \calO(r)\oplus m_B).
    \]
    Eventually when $\calA$ halts with bit $\hat{b}$, if $\hat{b}=b$ then
    $\calB$ outputs $1$, and otherwise it outputs $0$. This completes
    the description of $\calB$.

    We first claim that 
    \[
        \Pr[\calB^{E(k,\cdot)}=1] = \Pr[\ExptCPA_{\Pi}(\calA)=1],
    \]
    because in this case $\calB$ exactly simulates the experiment
    for $\calA$. That is, every oracle query is processed \emph{exactly}
    as in $\ExptCPA_{\Pi}(\calA)$ (and in particular a random key and bit are
    chosen, and each query uses a uniform and independent $r$ in addition to
    following the algorithm). Finally the output bit of $\calB$ is determined
    in exactly the same way.

    We next claim that
    \[
        \Pr[\calB^{f(\cdot)}=1] = \frac{1}{2} - \Col(2^\ell,q)
    \]
    where $q$ is the number of oracle queries issued by $\calA$.  Intuitively,
    with this oracle, $\calB$ is using a ``fresh one-time pad'' to mask every
    query response \emph{as long the $r$ does not repeat}. To formalize
    this a little bit,  let $R$ be the event that $\calB$ chooses the
    same value of $r$ in two different queries. Then $\Pr[R] = \Col(2^\ell,q)$.
    Writing $\bar{R}$ for ``not $R$'' and using the law of total probability,
    \begin{align*}
        \Pr[\calB^{f(\cdot)}=1] 
        & = \Pr[\calB^{f(\cdot)}=1|R]\Pr[R] 
             + \Pr[\calB^{f(\cdot)}=1|\bar{R}]\Pr[\bar{R}]  \\
        & \leq \Pr[R] 
             + \Pr[\calB^{f(\cdot)}=1|\bar{R}] \\
        & \leq \Col(2^\ell,q) + \frac{1}{2}. 
    \end{align*}
    The first inequality follows because we just omit some factors which are
    both probabilities and thus at most $1$. In the second inequality,
    we use $\Pr[R]=\Col(2^\ell,q)$ for the first part. To complete this
    we need to justify
    \begin{align*}
        \Pr[\calB^{\bof(\cdot)}=1|\bar{R}] = \frac{1}{2}. 
    \end{align*}
    Actually doing this completely rigorously is tricky. The key observation
    is that, if $R$ does not happen, that means all of queries are processed
    with unique values of $r$. Call them $r_1,\ldots,r_q$. Then the $i$-th
    query is answered with
    \[
        (r_i, \bof(r_i)\oplus m_b).
    \]
    But, conditioned on $R$ not happening,
    $\bof(r_1),\bof(r_2),\ldots,\bof(r_q)$ are all uniform and independent
    random variables on $\bits^\ell$. Thus each query completely masks out
    the message $m_b$, and hence are independent of $b$. Since every query
    response is independent of $b$, the adversary $\calA$ produces an output
    independent of $b$. Thus the probability that $\bhat=b$ is $1/2$, and
    this is the probability that $\calB$ outputs $1$.

    Finally, the rest is calculation:
    \begin{align*}
        \AdvPRF{E}{\calB} 
        & = |\Pr[\calB^{E(\bK,\cdot)}=1] - \Pr[\calB^{\bof(\cdot)}=1]| \\
        & \geq  \Pr[\calB^{E(\bK,\cdot)}=1] - \Pr[\calB^{\bof(\cdot)}=1] \\
        & =  \Pr[\ExptCPA_{\Pi}(\calA)=1] - \Pr[\calB^{\bof(\cdot)}=1] \\
        & \geq  \Pr[\ExptCPA_{\Pi}(\calA)=1] - \frac{1}{2} - \Col(2^\ell,q).
    \end{align*}
    The proof is completed by rearranging the final inequality.
\end{proof}

\begin{exercise}
    Define a stateful version of $\Pi$ by having $\Enc(k,m,s)$ output
    a ciphertext $(s,E(k,s)\oplus m)$ and updated state $s' = s+1$.
    (The initial state is $s=0^\ell$.) Adapt the proof to this case -- The
    result should be stronger, without the $\Col(2^\ell,q)$ term!
\end{exercise}

\end{document}

