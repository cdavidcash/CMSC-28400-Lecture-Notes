\documentclass[11pt]{article}
\usepackage{fullpage}
\usepackage{amsthm}

\usepackage{amsthm,amsmath,amsfonts,amssymb,amstext,enumitem}
\usepackage{latexsym,ifthen,url,rotating,graphicx}
\usepackage{listings}
\usepackage{tikz}
\usetikzlibrary{arrows,shapes,positioning,fit}
\usepackage{graphicx}
\usepackage[font=small,labelfont=bf]{caption}



% --- -----------------------------------------------------------------
% --- Document-specific definitions.
% --- -----------------------------------------------------------------
\lstset{
    columns=fixed,
    literate={—}{{---}}1 {…}{{...}}1
}

\newcommand{\todo}[1]{{\color{red}[TODO:{#1}]}}

\newtheorem{problem}{Problem}
\newtheorem{corollary}{Corollary}
\newtheorem{fact}{Fact}
\newtheorem{exercise}{Exercise}
\newtheorem{theorem}{Theorem}
\newtheorem{definition}{Definition}
\newtheorem{notation}{Notation}
\newtheorem{lemma}{Lemma}
\newtheorem{example}{Example}

\newcommand{\getsr}
  {{\:\stackrel{\raisebox{-2pt}{${\scriptscriptstyle \hspace{0.2em}\$}$}}
   {\leftarrow}\:}}
\newcommand{\points}[1]{\textbf{({#1} pts)}}

\newcommand{\fn}{\footnotesize}
\newcommand{\Colon}{\ : \ }
\newcommand{\st}{\mathsf{state}}
\newcommand{\msgs}{\mathcal{M}}
\newcommand{\ctxts}{\mathcal{C}}
\newcommand{\keys}{\mathcal{K}}
\newcommand{\rands}{\mathcal{R}}
\newcommand{\states}{\mathcal{S}}
\newcommand{\kg}{\mathcal{K}}
\newcommand{\Enc}{\mathsf{Enc}}
\newcommand{\Dec}{\mathsf{Dec}}
\newcommand{\MAC}{\mathrm{MAC}}
\newcommand{\RMAC}{\mathrm{RMAC}}

\newcommand{\pk}{pk}
\newcommand{\sk}{sk}

\newcommand{\calD}{\mathcal{D}}
\newcommand{\calA}{\mathcal{A}}
\newcommand{\calB}{\mathcal{B}}
\newcommand{\AES}{\mathsf{AES}}

\newcommand{\algorithm}[1]{\textbf{Alg} {#1}}

\newcommand{\calO}{\mathcal{O}}

\newcommand{\dlog}{\mathrm{dlog}}

\newcommand{\Adv}{\mathbf{Adv}}
\newcommand{\AdvPRF}[2]{\Adv^{\mathrm{prf}}_{#1}({#2})}
\newcommand{\AdvPRG}[2]{\Adv^{\mathrm{prg}}_{#1}({#2})}
\newcommand{\AdvCPA}[2]{\Adv^{\mathrm{cpa}}_{#1}({#2})}
\newcommand{\AdvCCA}[2]{\Adv^{\mathrm{cca}}_{#1}({#2})}
\newcommand{\AdvKR}[2]{\Adv^{\mathrm{kr}}_{#1}({#2})}
\newcommand{\AdvCKR}[2]{\Adv^{\mathrm{ckr}}_{#1}({#2})}
\newcommand{\AdvRMR}[2]{\Adv^{\mathrm{rmr}}_{#1}({#2})}
\newcommand{\AdvCR}[2]{\Adv^{\mathrm{cr}}_{#1}({#2})}
\newcommand{\AdvUFCMA}[2]{\Adv^{\textrm{uf{-}cma}}_{#1}({#2})}
\newcommand{\AdvDL}[2]{\Adv^{\mathrm{dl}}_{#1}({#2})}

\newcommand{\Exp}{\mathbf{Exp}}
\newcommand{\ExpOW}[1]{\Exp^{\mathrm{ow}}({#1})}
\newcommand{\ExpCKR}[2]{\Exp^{\mathrm{ckr}}_{#1}({#2})}
\newcommand{\ExpRMR}[2]{\Exp^{\mathrm{rmr}}_{#1}({#2})}

\newcommand{\concat}{{\,\|\,}}
\newcommand{\xor}{\oplus}
\newcommand{\bits}{\{0,1\}}

\newcommand{\tcolh}{T^{\mathrm{col}}_h}
\newcommand{\tcolH}{T^{\mathrm{col}}_{H^2}}
\newcommand{\Hcomb}{H^{1\|2}}
\newcommand{\Hxor}{H^{1\oplus2}}

\newcommand{\EXP}{\textrm{EXP}}
\newcommand{\MODEXP}{\textrm{MOD{-}EXP}}
\newcommand{\ADD}{\textrm{ADD}}
\newcommand{\MULTIMODEXP}{\textrm{MULTI{-}MOD{-}EXP}}
\newcommand{\MUL}{\textrm{MUL}}
\newcommand{\MOD}{\textrm{MOD}}

\newcommand{\GG}{\mathbb{G}}
\newcommand{\ZZ}{\mathbb{Z}}

\newcommand{\bK}{\mathbf{K}}
\newcommand{\bof}{\mathbf{f}}
\newcommand{\bU}{\mathbf{U}}
\newcommand{\bM}{\mathbf{M}}
\newcommand{\bC}{\mathbf{C}}

\newcommand{\rvrange}{\mathcal{R}}
\newcommand{\rspace}{\mathcal{C}}

\newcommand{\hatalpha}{\hat{\alpha}}
\newcommand{\hatb}{\hat{b}}

\newcommand{\barm}{\overline{m}}

\newcommand{\otp}{\mathrm{OTP}}
\newcommand{\des}{\mathrm{DES}}
\newcommand{\twodes}{\mathrm{2DES}}
\newcommand{\threedes}{\mathrm{3DES}}
\newcommand{\threedestwo}{\mathrm{3DES2}}
\newcommand{\aes}{\mathrm{AES}}
\newcommand{\pad}{\mathsf{pad}}
\newcommand{\unpad}{\mathsf{unpad}}
\newcommand{\Func}{\mathrm{Func}}


\newcommand{\Img}{\mathrm{Im}}

\newcommand{\Expt}{\mathbf{Expt}}
\newcommand{\ExptCPA}{\mathbf{Expt}^{\mathrm{cpa}}}
\newcommand{\ExptCCA}{\mathbf{Expt}^{\mathrm{cca}}}
\newcommand{\ExptOTCPA}{\mathbf{Expt}^{\mathrm{1\mbox{-}cpa}}}
\newcommand{\ExptOTCPAone}{\mathbf{Expt}^{\mathrm{1\mbox{-}cpa\mbox{-}1}}}
\newcommand{\ExptOTCPAzero}{\mathbf{Expt}^{\mathrm{1\mbox{-}cpa\mbox{-}0}}}
\newcommand{\AdvOTCPA}[2]{\Adv^{\mathrm{1\mbox{-}cpa}}_{#1}({#2})}
\newcommand{\ExptCPAone}{\mathbf{Expt}^{\mathrm{cpa\mbox{-}1}}}
\newcommand{\ExptCPAzero}{\mathbf{Expt}^{\mathrm{cpa\mbox{-}0}}}

\newcommand{\LR}{\mathrm{LR}}

\newcommand{\Piotp}{\Pi_\mathrm{otp}}
\newcommand{\Encotp}{\Enc_\mathrm{otp}}
\newcommand{\Decotp}{\Dec_\mathrm{otp}}
\newcommand{\bhat}{\hat{b}}
\newcommand{\dict}{\mathtt{dict}}
\newcommand{\Col}{\mathsf{Col}}
% --- -----------------------------------------------------------------
% --- Lecture notes formatting macros
% --- -----------------------------------------------------------------

%
% The following commands set up the lecnum (lecture number)
% counter and make various numbering schemes work relative
% to the lecture number.
%
\newcounter{lecnum}
%\renewcommand{\thepage}{\thelecnum-\arabic{page}}
\renewcommand{\thesection}{\thelecnum.\arabic{section}}
\renewcommand{\theexercise}{\thelecnum.\arabic{exercise}}
\renewcommand{\theexample}{\thelecnum.\arabic{example}}
\renewcommand{\thedefinition}{\thelecnum.\arabic{definition}}
\renewcommand{\theequation}{\thelecnum.\arabic{equation}}
\renewcommand{\thefigure}{\thelecnum.\arabic{figure}}
\renewcommand{\thefact}{\thelecnum.\arabic{fact}}
\renewcommand{\thetable}{\thelecnum.\arabic{table}}


%
% The following macro is used to generate the header.
%
\newcommand{\lecture}[2]{
   %\pagestyle{myheadings}
   %\thispagestyle{plain}
   \newpage
   \setcounter{lecnum}{#1}
   \setcounter{page}{1}
   \noindent
   \begin{center}
   \framebox{
      \vbox{\vspace{2mm}
    \hbox to 6.28in { {\bf CMSC 28400 Introduction to Cryptography
                        \hfill Autumn 2019} }
       \vspace{4mm}
       \hbox to 6.28in { {\Large \hfill #2 \hfill} }
       \vspace{2mm}
       \hbox to 6.28in { {\it Instructor: David Cash} \hfill }
      \vspace{2mm}}
   }
   \end{center}
   %\markboth{Lecture #1: #2}{Lecture #1: #2}
   \vspace*{4mm}
}





% --- -----------------------------------------------------------------
% --- The document starts here.
% --- -----------------------------------------------------------------
\begin{document}
%\lecture{**LECTURE-NUMBER**}{**DATE**}{**LECTURER**}{**SCRIBE**}
\lecture{10}{Notes \#11: Chosen-Ciphertext Security for Encryption}

%\tableofcontents

The following definition is
motivated by the padding oracle attack against AES-CBC, and in general
by an adversary's ability to inject ciphertexts to induce reactions from
the honest parties.
\begin{definition}
    Let $\Pi = (\Enc,\Dec)$ be a randomized encryption scheme with key-space
    $\keys$, message-space $\msgs$, randomness-space $\rands$, and
    ciphertext-space $\ctxts$.  We assume that the message-space $\msgs$ is a
    set of bit strings, i.e.  $\msgs\subseteq\bits^*$.  Let $\calA$ be an
    algorithm. Define algorithm $\ExptCCA_\Pi(\calA)$ as
    \begin{center}
    \begin{tabular}{c}
        \begin{minipage}{2in}\begin{tabbing}
            123\=123\=\kill
            \underline{\algorithm{$\ExptCCA_\Pi(\calA)$}} \\[2pt]
            \fn01 \> Pick $k\getsr \keys, b\getsr \bits$\\
            \fn02 \> Run $\calA^{\LR_{k,b}(\cdot,\cdot),\Dec(k,\cdot)}$, where
            the first oracle is given below. Eventually $\calA$ halts with
            output $\hatb$\\
            \fn03 \> If $\calA$ every queried $\Dec(k,\cdot)$ at a ciphertext
            $c$ previously output by $\LR_{k,b}(\cdot,\cdot)$: Output $\bot$. \\
            \fn04\> If $\hatb = b$: Output 1\\
            \fn05 \> Else: Output 0\\
            \\
            \underline{Oracle $\LR_{k,b}(m_0,m_1)$} \\
            \> If $m_0,m_1$ are not the same length: Return $\bot$\\
            \> Pick $r \getsr \rands$\\
            \> Compute $c \gets \Enc(k,m_b,r)$\\
            \> Return $c$
        \end{tabbing}\end{minipage}
    \end{tabular}
    \end{center}
    Define the \emph{CCA advantage of $\calA$ against $\Pi$} as
    \[
        \AdvCCA{\Pi}{\calA} =
        \left|\Pr[\ExptCCA_\Pi(\calA) = 1] - \frac{1}{2}\right|
    \]

    The definition for the stateful version of the scheme
    is the same, except the first oracle works as follows:
    \begin{center}
        \begin{tabular}{c}
            \begin{minipage}{2in}\begin{tabbing}
                123\=123\=\kill
                \underline{Oracle $\LR_{k,b}(m_0,m_1)$} \\
                \> If $m_0,m_1$ are not the same length: Return $\bot$\\
                \> Compute $(c,s') \gets \Enc(k,m_b,s)$\\
                \> Overwrite the current state $s$ with $s'$ \\
                \> Return $c$
            \end{tabbing}\end{minipage}
        \end{tabular}
    \end{center}
\end{definition}

Note that the game is easy to win if we omit the rule on line $03$;
We leave this as an exercise to check.
\begin{exercise}
    Consider a version of the previous definition with line $03$ deleted.
    Assume $\Pi$ has more than one allowed message, i.e.  $|\msgs| > 1$. Give
    an efficient adversary $\calA$ such that
    \[
        \AdvCCA{\Pi}{\calA} = 1/2.
    \]
\end{exercise}

Building encryption from a block cipher that resists chosen-ciphertext attacks
is considerably harder than chosen-plaintext attacks. The following example
is meant to highlight the problem.

\begin{example}
Let $E:\bits^n\times\bits^\ell\to\bits^\ell$ be a block cipher. 
Define a randomized encryption encryption scheme 
    $\Pi=(\Enc,\Dec)$ with key-space $\bits^n$, message-
    and randomness-spaces $\msgs=\rands=\bits^\ell$, and ciphertext-space
    $\ctxts=\bits^\ell\times\bits^\ell$ by
    \[
        \Enc(k,m,r)=(r,E(k,r)\oplus m)
    \]
    and $\Dec(k,(r,c)) = E(k,r)\oplus c$.

    Previously we showed that breaking the CPA-security of $\Pi$ required
    breaking the PRF-security of $E$. We now give an efficient adversary
    $\calA$ that strongly breaks the CCA-security of $\Pi$, satisfying
    \[
        \AdvCCA{\Pi}{\calA} = 1/2.
    \]
    This shows the construction is not at all secure against chosen-ciphertext
    attack.
    The adversary works follows:

    \begin{center}
        \begin{tabular}{c}
            \begin{minipage}{2in}\begin{tabbing}
                123\=123\=\kill
                \underline{Adversary $\calA^{\LR_{k,b}(m_0,m_1),\Dec(k,\cdot)}$} \\[2pt]
                \> Let $m_0\neq m_1\in\msgs$, with $m_0\neq 0^\ell$. 
                Query $(r,c) \gets \LR_{k,b}(m_0,m_1)$.\\
                \> Let $c' = c \oplus m_0$. Query $m' \gets \Dec(k,(r,c'))$\\
                \> If $m' = 0^\ell$: Output $0$\\
                \> Else: Output $1$.
            \end{tabbing}\end{minipage}
        \end{tabular}
    \end{center}

    Let us check that the adversary always win $\ExptCCA_\Pi(\calA)$.
    First, $\calA$ will never auto-lose the game, because $(r,c)$ is never
    queried to the decryption oracle (only $(r,c')$ is, but $(r,c')\neq (r,c)$
    because $c'\neq c$, which is where we use that $m_0\neq 0^\ell$). Now
    suppose $b=0$, so $(r,c) = (r,E(k,r)\oplus m_0)$. Then the ciphertext
    $(r,c')$ will decrypt as
    \[
        c' \oplus E(k,r) 
        = c\oplus m_0 \oplus E(k,r) 
        = E(k,r)\oplus m_0 \oplus m_0 \oplus E(k,r) 
        = 0^\ell,
    \]
    and $\calA$ will output $0$. Otherwise, if $b=1$, $(r,c')$ will decrypt
    as
    \[
        c' \oplus E(k,r) 
        = c\oplus m_0 \oplus E(k,r) 
        = E(k,r)\oplus m_1 \oplus m_0 \oplus E(k,r) 
        = m_1\oplus m_0 \neq 0^\ell,
    \]
    because $m_0\neq m_1$. Thus $\calA$ will output $1$. This shows that
    $\calA$ always wins and thus has the desired advantage.
\end{example}
Intuitively, the problem is \emph{malleability of ciphertexts}: The adversary
$\calA$, without knowing the key, is able to modify a ciphertext into a new
ciphertext containing a message related to the original one. This new
ciphertext can be submitted for decryption (because it is different), and
the response can be used to win the experiment.

\begin{exercise}
    Let $\Pi$ be an encryption scheme. Show that for all $\calA$ there exists
    a $\calB$ running in the same time such that
    \[
        \AdvCPA{\Pi}{\calA} \leq \AdvCCA{\Pi}{\calB}.
    \]
    How are CCA and CPA security related?
\end{exercise}

\begin{exercise}
    Let $\Pi$ be any of the block ciphers we have seen so far (CTR and CCA in
    their stateful and randomized modes). For each one, find an efficient
    $\calA$ such that
    \[
        \AdvCCA{\Pi}{\calA} = 1/2.
    \]
\end{exercise}

\end{document}

